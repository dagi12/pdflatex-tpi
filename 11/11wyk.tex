\documentclass[a4paper,12pt]{article}

%% Language and font encodings
\usepackage[T1]{fontenc}
\usepackage[polish]{babel}
\usepackage[utf8]{inputenc}
\usepackage{lmodern}
\selectlanguage{polish}

%% Sets page size and margins
%\usepackage[a4paper,top=2cm,bottom=2cm,left=2cm,right=4cm,asymmetric]{geometry}
\usepackage{geometry}


%% Useful packages
\usepackage[fleqn]{amsmath}%[fleqn]
\usepackage{xfrac} %\sfrac{}{}

\usepackage{titlesec}%titles
\titlelabel{\thetitle.\quad}
\let\savenumberline\numberline
\def\numberline#1{\savenumberline{#1.}}
\usepackage{etoolbox}%dots in TOC
\makeatletter
\patchcmd{\l@section}
{\hfil}
{\leaders\hbox{\normalfont$\m@th\mkern \@dotsep mu\hbox{.}\mkern \@dotsep mu$}\hfill}
{}{}
\makeatother

\usepackage{caption,graphicx}
\usepackage{float}
\usepackage{sidecap}
\usepackage[colorinlistoftodos]{todonotes}
\usepackage[colorlinks=true, allcolors=blue]{hyperref}

\usepackage{tikz}
\usepackage{tikz-qtree}
\usetikzlibrary{trees}
\usetikzlibrary{arrows,positioning,shapes,fit,calc,decorations.pathreplacing}
\usetikzlibrary{graphs}
\usetikzlibrary{graphs.standard}
\usepackage{forest}
\usepackage{tikzscale}
\usepackage{pgfgantt} %Diagramy gantta http://bay.uchicago.edu/CTAN/graphics/pgf/contrib/pgfgantt/pgfgantt.pdf
\usepackage{pgf}
\usepackage{caption}

\usepackage{fancybox}
\usepackage{listings}

\usepackage[colorinlistoftodos]{todonotes} %http://mirror.unl.edu/ctan/macros/latex/contrib/todonotes/todonotes.pdf

\usepackage{array,longtable}

\newcommand\floor[1]{\lfloor#1\rfloor} %PODŁOGA -> \floor
\newcommand\ceil[1]{\lceil#1\rceil} %SUFIT -> \ceil

\usepackage{fancyhdr}
\pagestyle{fancy}
\rhead{\thepage}
\lhead{\leftmark}
\rfoot{\thepage}
\lfoot{\rightmark}

%http://tex.stackexchange.com/questions/64170/which-package-to-use-for-writing-algorithms
\usepackage{algorithm}% http://ctan.org/pkg/algorithms
\usepackage{algpseudocode}% http://ctan.org/pkg/algorithmicx
\newcommand{\var}[1]{{\ttfamily#1}}% variable

\algnewcommand\algorithmicforeach{\textbf{for each}} %Algorithm foreach
\algdef{S}[FOR]{ForEach}[1]{\algorithmicforeach\ #1\ \algorithmicdo}

\usepackage{amsthm}
\usepackage[inline]{enumitem} %enumerations
\usepackage{multicol}

\theoremstyle{definition}%~ %%% <-  Note that space!
\newtheorem{lemma}{Lemat} %\begin{lemma} ... \end{lemma} LEMAT(?)
\newtheorem{remark}{Wniosek}%\begin{remark} ... \end{remark} WNIOSEK 
%\newtheorem*{remark*}{Wniosek}%\begin{remark} ... \end{remark} WNIOSEK bez liczby
\newtheorem{theorem}{Twierdzenie}%\begin{theorem} ... \end{theorem}
\newtheorem{fact}{Fakt} %\begin{fact} ... \end{fact}
\newtheorem*{fact*}{Fakt} %\begin{fact*} ... \end{fact*} Fakt
\newtheorem*{observation*}{Obserwacja}

\newtheorem{example}{Przykład}
\newtheorem*{example*}{Przykład} %\begin{example} ... \end{example}
\theoremstyle{definition}
\newtheorem{definition}{Definicja}%\begin{definition}{} ... \end{definition}
%\newtheorem*{definition*}{Definicja}
%\newtheorem*{hipoterm*}{Hipoteza}%\begin{hipoterm*}[] ... \end{hipoterm*}
\newtheorem{hipoterm}{Hipoteza}%\begin{hipoterm*}[] ... \end{hipoterm*}
\theoremstyle{problem}
\newtheorem{problem}{Problem}%\begin{problem}{} ... \end{problem}
\newtheorem*{problem*}{Problem}

\let\originalforall=\forall%FORALL
\renewcommand{\forall}{\mathop{\vcenter{\hbox{\Large$\originalforall$}}}}
\let\originalexists=\exists%EXISTS
\renewcommand{\exists}{\mathop{\vcenter{\hbox{\Large$\originalexists$}}}}

\usepackage{cancel} %skreślenie równania \xcancel{...} \cancel{} lub \bcancel{}

\usepackage{amsfonts}

\usepackage{comment}

\usepackage{xcolor,colortbl}
\usepackage{multirow}

%\usepackage{wrapfig} %wrap text around figure

\usepackage{pdfpages}%\includepdf{file}

\usepackage{etoolbox}
\let\bbordermatrix\bordermatrix
\patchcmd{\bbordermatrix}{8.75}{4.75}{}{}
\patchcmd{\bbordermatrix}{\left(}{\left[}{}{}
\patchcmd{\bbordermatrix}{\right)}{\right]}{}{}
%\bbordermatrix{}

\allowdisplaybreaks

\title{Struktury Dyskretne - Notatki}
\author{Piotr Parysek\\
\href{mailto:piotr.parysek@outlook.com}{piotr.parysek@outlook.com} }
\date{\today}

\begin{document}
\maketitle

\tableofcontents
\section[Wykład 11: 25-V-2017 - Temat: Teoria kodów II]{Temat: Teoria kodów II}
\subsection{Definicje}
$$Q=\{0,1,\dots,q-1\}=F_q$$
przyjmujemy, że $q$ to liczba pierwsza
$$c\subseteq Q^n$$
$$r(C)=\min_{\begin{matrix}
\bar{x},\bar{y}\in C\\
\bar{x}\neq\bar{y}
\end{matrix}}$$
Odległość Hamminga - czyli na ilu miejscach $\bar{x}$ różni się od $\bar{y}$

$[n,k]$-kod - podprzestrzeń liniowa $Q^n$ o wymiarze $k$

\begin{example*}[Przykład wykorzystywany w reszcie wykładu]
\begin{align*}
&C=\{(0,0,0,0),(0,1,1,1),(0,2,2,2),(1,0,2,1),(2,0,1,2),(1,1,0,2),(2,2,0,1),(1,2,1,0)\}\\
&Q=\mathbb{F}_3\\
&G=\begin{bmatrix}
0&0&2&1\\0&1&1&1
\end{bmatrix}\\
&G_2=\begin{bmatrix}
1&1&0&2\\2&0&1&2
\end{bmatrix}
\end{align*}
\end{example*}

\begin{definition}[Macierz parzystości]
Macierz sprawdzania parzystości (dalej nazywana \textbf{Macierzą parzystości}) $H$ to macierz, której wierszami jest $n-k$ niezależnych wektorów, z których każdy jest ortogonalny do każdego wiersza macierzy $G$.
\end{definition}
\begin{example*}
\begin{align*}
&H_1=\begin{bmatrix}
1&0&2&1\\0&1&1&1
\end{bmatrix}\\
&GH^T=[ {\Huge 0}]\\
&\begin{bmatrix}
1&0&2&1\\0&1&1&1
\end{bmatrix}\begin{bmatrix}
0&1\\1&0\\1&2\\1&1
\end{bmatrix}=\begin{bmatrix}
(0+0+2+1)\%3 & (1+0+4+1)\%3\\
(0+1+1+1)\%3 & (0+0+2+1)\%3 
\end{bmatrix}=\begin{bmatrix}
0&0\\
0&0
\end{bmatrix}
\end{align*}
Gdy macierz $G$ jest postaci (postać normalna):
$$G=\begin{bmatrix}
{\Huge I_k}& {\Huge P}
\end{bmatrix}$$
to
$$\begin{bmatrix}
{\Huge -P^T} & {\Huge I_{n-k}}
\end{bmatrix}$$ jest (jedną z) macierzy parzystości
\begin{align*}
&G_1=\begin{bmatrix}
1&0&2&1\\0&1&1&1
\end{bmatrix} & H_1=\begin{bmatrix}
1&2&1&0\\2&2&0&1
\end{bmatrix}\\
&\mathbb{F}_5\\
&G=\begin{bmatrix}
1&0&0&2&3\\
0&1&0&1&1\\
0&0&1&2&1
\end{bmatrix} & H=\begin{bmatrix}
3&4&3&1&0\\
2&4&4&0&1
\end{bmatrix}\\
&H=\left[{\huge -P^T\ I_{n-k}}\right]=\begin{bmatrix}
5-2 & 5-1 & 5-2 & 1 & 0\\
5-3 & 5-1 & 5-1 & 0 & 1
\end{bmatrix}\\
&G=\left[{\huge I_k\ P}\right]=\begin{bmatrix}
1 & 0 & 0 & 5-3 & 5-2\\
0 & 1 & 0 & 5-4 & 5-4\\
0 & 0 & 1 & 5-3 & 5-4
\end{bmatrix}
\end{align*}
\end{example*}

\begin{fact}
$$\bar{x}\in C\Leftrightarrow H\bar{x}=[0]$$
\end{fact}
\begin{example*}
Mając macierz parzystości $H=\begin{bmatrix}
1&2&1&0\\2&2&0&1
\end{bmatrix}$ czy wektor $\bar{x}=2120$ należy do $C$?

\begin{align*}
&H\bar{x}=[0]\\
&\begin{bmatrix}
1&2&1&0\\2&2&0&1
\end{bmatrix}\begin{bmatrix}
2\\1\\2\\0
\end{bmatrix}\Leftrightarrow \begin{bmatrix}
(2+2+2+0)\%3\\
(4+2+0+0)\%3
\end{bmatrix}=\begin{bmatrix}
0\\0
\end{bmatrix}
\end{align*}
\textbf{Odpowiedź: }tak wektor $\bar{x}$ należy do kodu $C$.

Czy wektor $\bar{y}=2200$ należy do $C$?
\begin{align*}
&H\bar{y}=[0]\\
&\begin{bmatrix}
1&2&1&0\\2&2&0&1
\end{bmatrix}\begin{bmatrix}
2\\2\\0\\0
\end{bmatrix}\Leftrightarrow \begin{bmatrix}
(2+4+0+0)\%3\\
(4+4+0+0)\%3
\end{bmatrix}=\begin{bmatrix}
0\\2
\end{bmatrix}
\end{align*}
\textbf{Odpowiedź: }NIENIU!
\end{example*}
\begin{problem}[Jak znaleźć wektor należący do $C$?]
\begin{align*}
H\bar{x}\neq \bar{0}& H\bar{x}=a\neq \bar{0}\\
&H\bar{y} =a\neq\bar{0}\\
&H(\bar{x}-\bar{y})=H\bar{x}-H\bar{y}=\bar{0}\\
&\bar{x}-\bar{y}\in C
\end{align*}
czyli trzeba znaleźć $\bar{y}$ taki, że $H\bar{y}=a$ i $\bar{y}$ ma jak najwięcej współrzędnych zerowych.
\end{problem}
\begin{align*}
\bar{x}=2200\\
\bar{y}=? &\bar{y}=0002\\
&\bar{x}-\bar{y}=2200-0002=2201
\end{align*}
\textbf{Pytanie:} Jak skonstruować kody doskonałe?
\begin{definition}[Kod Hamminga]
Kodem Hamminga nazywamy $[n, n-l, 3]$ kod nad ciałem $F_q$ taki, że: $$n=\frac{q^l-1}{q-1}$$

Macierz parzystości kodów Hamminga można wygenerować w prosty sposób: jako kolumny macierzy należy wziąć $n$ wektorów \textbf{parami niezależnych}.
\end{definition}
\begin{fact}
Kod Hamminga jest kodem doskonałym.
\end{fact}
\begin{example*}
\begin{align*}
q=5 & n=6 &l=2 \\
n=\frac{q^l-1}{q-1} & \rightarrow & 6=\frac{5^2-1}{5-1} &\rightarrow &6=\frac{24}{6}
\end{align*}
$$H=\begin{bmatrix}
1&0&1&2&1&2\\0&1&2&3&3&2
\end{bmatrix}$$

\end{example*}


%-------------------- ĆWICZENIA --------------------
\end{document}
