\documentclass[a4paper,12pt]{article}

%% Language and font encodings
\usepackage[T1]{fontenc}
\usepackage[polish]{babel}
\usepackage[utf8]{inputenc}
\usepackage{lmodern}
\selectlanguage{polish}

%% Sets page size and margins
%\usepackage[a4paper,top=2cm,bottom=2cm,left=2cm,right=4cm,asymmetric]{geometry}
\usepackage{geometry}


%% Useful packages
\usepackage[fleqn]{amsmath}%[fleqn]
\usepackage{xfrac} %\sfrac{}{}

\usepackage{titlesec}%titles
\titlelabel{\thetitle.\quad}
\let\savenumberline\numberline
\def\numberline#1{\savenumberline{#1.}}
\usepackage{etoolbox}%dots in TOC
\makeatletter
\patchcmd{\l@section}
{\hfil}
{\leaders\hbox{\normalfont$\m@th\mkern \@dotsep mu\hbox{.}\mkern \@dotsep mu$}\hfill}
{}{}
\makeatother

\usepackage{caption,graphicx}
\usepackage{float}
\usepackage{sidecap}
\usepackage[colorinlistoftodos]{todonotes}
\usepackage[colorlinks=true, allcolors=blue]{hyperref}

\usepackage{tikz}
\usepackage{tikz-qtree}
\usetikzlibrary{trees}
\usetikzlibrary{arrows,positioning,shapes,fit,calc,decorations.pathreplacing}
\usetikzlibrary{graphs}
\usetikzlibrary{graphs.standard}
\usepackage{forest}
\usepackage{tikzscale}
\usepackage{pgfgantt} %Diagramy gantta http://bay.uchicago.edu/CTAN/graphics/pgf/contrib/pgfgantt/pgfgantt.pdf
\usepackage{pgf}
\usepackage{caption}

\usepackage{fancybox}
\usepackage{listings}

\usepackage[colorinlistoftodos]{todonotes} %http://mirror.unl.edu/ctan/macros/latex/contrib/todonotes/todonotes.pdf

\usepackage{array,longtable}

\newcommand\floor[1]{\lfloor#1\rfloor} %PODŁOGA -> \floor
\newcommand\ceil[1]{\lceil#1\rceil} %SUFIT -> \ceil

\usepackage{fancyhdr}
\pagestyle{fancy}
\rhead{\thepage}
\lhead{\leftmark}
\rfoot{\thepage}
\lfoot{\rightmark}

%http://tex.stackexchange.com/questions/64170/which-package-to-use-for-writing-algorithms
\usepackage{algorithm}% http://ctan.org/pkg/algorithms
\usepackage{algpseudocode}% http://ctan.org/pkg/algorithmicx
\newcommand{\var}[1]{{\ttfamily#1}}% variable

\algnewcommand\algorithmicforeach{\textbf{for each}} %Algorithm foreach
\algdef{S}[FOR]{ForEach}[1]{\algorithmicforeach\ #1\ \algorithmicdo}

\usepackage{amsthm}
\usepackage[inline]{enumitem} %enumerations
\usepackage{multicol}

\theoremstyle{definition}%~ %%% <-  Note that space!
\newtheorem{lemma}{Lemat} %\begin{lemma} ... \end{lemma} LEMAT(?)
\newtheorem{remark}{Wniosek}%\begin{remark} ... \end{remark} WNIOSEK 
%\newtheorem*{remark*}{Wniosek}%\begin{remark} ... \end{remark} WNIOSEK bez liczby
\newtheorem{theorem}{Twierdzenie}%\begin{theorem} ... \end{theorem}
\newtheorem{fact}{Fakt} %\begin{fact} ... \end{fact}
\newtheorem*{fact*}{Fakt} %\begin{fact*} ... \end{fact*} Fakt
\newtheorem*{observation*}{Obserwacja}

\newtheorem{example}{Przykład}
\newtheorem*{example*}{Przykład} %\begin{example} ... \end{example}
\theoremstyle{definition}
\newtheorem{definition}{Definicja}%\begin{definition}{} ... \end{definition}
%\newtheorem*{definition*}{Definicja}
%\newtheorem*{hipoterm*}{Hipoteza}%\begin{hipoterm*}[] ... \end{hipoterm*}
\newtheorem{hipoterm}{Hipoteza}%\begin{hipoterm*}[] ... \end{hipoterm*}
\theoremstyle{problem}
\newtheorem{problem}{Problem}%\begin{problem}{} ... \end{problem}
\newtheorem*{problem*}{Problem}

\let\originalforall=\forall%FORALL
\renewcommand{\forall}{\mathop{\vcenter{\hbox{\Large$\originalforall$}}}}
\let\originalexists=\exists%EXISTS
\renewcommand{\exists}{\mathop{\vcenter{\hbox{\Large$\originalexists$}}}}

\usepackage{cancel} %skreślenie równania \xcancel{...} \cancel{} lub \bcancel{}

\usepackage{amsfonts}

\usepackage{comment}

\usepackage{xcolor,colortbl}
\usepackage{multirow}

%\usepackage{wrapfig} %wrap text around figure

\usepackage{pdfpages}%\includepdf{file}

\usepackage{etoolbox}
\let\bbordermatrix\bordermatrix
\patchcmd{\bbordermatrix}{8.75}{4.75}{}{}
\patchcmd{\bbordermatrix}{\left(}{\left[}{}{}
\patchcmd{\bbordermatrix}{\right)}{\right]}{}{}
%\bbordermatrix{}

\allowdisplaybreaks

\title{Struktury Dyskretne - Notatki}
\author{Piotr Parysek\\
\href{mailto:piotr.parysek@outlook.com}{piotr.parysek@outlook.com} }
\date{\today}

\begin{document}
\maketitle

\tableofcontents
\section{Ćwiczenia 25-V-2017}

Wszędzie obowiązuje domyślne założenie, że wszystkie zmienne występujące w zagadnieniach programowania liniowego mogą przyjmować tylko wartości nieujemne.
\subsection{Ściąga}
\begin{description}
\item[$\beta $] $\rightarrow$ wielkość największego skojarzenia w grafie - ile jest skojarzeń.
\item[$\beta ^\star$] $\rightarrow$ maksimum funkcji $\sum _{e\in E(G)} x_e$ takie że $\sum _{e:v\in e} x_e\leq 1$ dla każdego $v \in V (G)$.
\item[$\gamma $] $\rightarrow$ minimalna liczba pokryciowa i najmniejsza liczba wierzchołków by mieć połączone ,,wszystkie'' krawędzie.
\item[$\gamma ^\star$] $\rightarrow$ minimum funkcji $\sum _{v\in V(G)} x_v$ takie że $\sum _{v:v\in e} x_v \geq 1$ dla każdego $e \in E(G)$.
\item[$\omega $] $\rightarrow$ liczba klikowa wyjaśniona na górze. 
\item[$\omega ^\star$] $\rightarrow$ maksimum funkcji $\sum _{v\in V(G)} x_v$ takie że $\sum _{v:v\in I} x_v \leq 1$ dla każdego zbioru niezależnego $I \subseteq V(G)$.
\item[$\chi $] $\rightarrow$ liczba chromatyczna - czyli ile kolorków muszą użyć by pomalować graf w taki sposób, aby żadne połączone dwa wierzchołki nie miały tego samego koloru.
\item[$\chi ^\star$] $\rightarrow$ minimum funkcji $\sum _{I\subseteq V(G)} x_I$ takie że $\sum _{I:v\in I} x_I \geq 1$ dla każdego $v \in V(G)$, przy czym w obu sumowaniach występują tylko zbiory niezależne. 
\end{description}

\subsection{Zadania domowe A}
\paragraph{A1} Sformułuj zagadnienia dualne do podanych poniżej. Nie rozwiązuj problemów.\\
\begin{minipage}{.5\textwidth}
\textbf{Problem 1} Zminimalizować funkcję
$$f(x, y, z) = z - x + y$$
dla zmiennych $x, y, z$ spełniających warunki:
$$\left\{\begin{matrix}
-x &+ 2y &- z &\geq -1\\
&-y &+z &\geq \sfrac{3}{2}
\end{matrix}\right.$$
\begin{align*}
&\begin{matrix}
y_1: & -x&+2y&-z&\geq 1\\
y_2:& &-y&+z&\geq\frac{3}{2}
\end{matrix}\\
&g(y_1,y_2)=-1y_1+\frac{3}{2}y_2\\
&\left\{\begin{matrix}
-y_1 &&\leq -1\\
2y_1&-y_2&\leq 1\\
-y_1&+y_2&\leq 1
\end{matrix}\right.\\
&\min _{x,y,z}f(x,y,z)=\max _{y_1,y_2} g(y_1,y_2)
\end{align*}
\end{minipage}%
\begin{minipage}{.5\textwidth}
\textbf{Problem 2} Zmaksymalizować funkcję
$$f(x, y, z, w) = x - y$$
dla zmiennych $x, y, z$, w spełniających warunki:
$$\left\{\begin{matrix}
x +&&& w &\leq 1\\
x -& y &&- \sfrac{1}{4}w &\leq 0\\
x -&2y +& z +&w &\leq 2
\end{matrix}\right.$$
\begin{align*}
&\begin{matrix}
y_1: & x &+ 	&	& w	&\leq  1\\
y_2: & x &- y  	&	&- \sfrac{1}{4} w &\leq 0\\
y_3: & x &- 2y 	&+z	&+w	&\leq 2
\end{matrix}\\
&g(y_1,y_2,y_3)=y_1+2y_3\\
&\left\{\begin{matrix}
y_1 &+ y_2 &+ y_3 &\geq 1\\
&-y_2 & -2y_2 &\geq -1\\
&&y_3 &\geq 0\\
y_1 &- \sfrac{1}{4} y_2 &+ y_3 &\geq 0
\end{matrix}\right.\\
&\max _{x,y,z,w} f(x,y,z,w)=\min _{y_1,y_2,y_3} (y_1,y_2,y_3)
\end{align*}
\end{minipage}

\paragraph{A2} Czy drugi problem jest dualny do pierwszego problemu, sformułowanego jako zagadnienie programowania liniowego? Rozwiąż (zdroworozsądkowo) oba problemy.

\begin{minipage}{.5\textwidth}
\textbf{Problem 1} Zmaksymalizować funkcję
$$f(x, y, z) = x - 2y$$
dla zmiennych $x, y, z$ spełniających warunki:
$$\left\{\begin{matrix}
x &+ y &&\leq 1\\
&-y &+ z &\leq \frac{1}{2}
\end{matrix}\right.$$
\begin{align*}
&\begin{matrix}
y_1: & x_1 &+x_2& &\leq 1\\
y_2: & &-x_2&+x_3 &\leq \sfrac{1}{2}
\end{matrix}\\
&g(y_1,y_2)=y_1+\frac{1}{2}y_2\\
&\left\{\begin{matrix}
y_1&&\geq 1\\
y_2&-y_2&\geq -2
\end{matrix}\right.\\
&\max _{x,y,z}f(x,y,z)=\min _{y_1,y_2}g(y_1,y_2)
\end{align*}
\end{minipage}%
\begin{minipage}{.5\textwidth}
\textbf{Problem 2} Zminimalizować funkcję
$$g(x, y) = x - \frac{1}{2}y$$
dla zmiennych $x, y$ spełniających warunki:
$$\left\{\begin{matrix}
x &&\geq 1\\
x &- y &\geq -2\\
&y &\geq 0
\end{matrix}\right.$$
\begin{align*}
&\begin{matrix}
y_1: & x_1 & &\geq 1\\
y_2: & x_1 &-x_2 &\geq -2\\
y_3: & & x_2 &\geq 0
\end{matrix}\\
&f(y_1,y_2,y_3)=y_1-2y_2\\
&\left\{\begin{matrix}
y_1	&+y_2	&		&\leq 1\\
&-y_2	&+y_3	&\leq \frac{1}{2}
\end{matrix}\right.\\
&\min _{x,y}g(x,y)=\max _{y_1,y_2,y_3}f(y_1,y_2,y_3)
\end{align*}
\end{minipage}




\paragraph{A3} Dla podanych grafów (z pewnymi funkcjami zdefiniowanymi na zbiorze wierzchołków lub krawędzi) uzasadnij
prawdziwość podanych oszacowań na parametry $\beta ^\star, \gamma ^\star, \chi ^\star, \omega ^\star$. (Parametry te zdefiniowane były na wykładzie.)

\begin{enumerate}[label=\alph*)]
\item $\beta ^\star \geq 2$ 
\begin{figure}[H]
\centering
\begin{tikzpicture}[shorten >=1pt, auto, node distance=3cm, ultra thick,main node/.style={circle,draw,minimum size=.4cm,inner sep=0pt]}]%fill=black,
\begin{scope}%[every node/.style={font=\sffamily\Large\bfseries}]%[main node]
\node[main node] (v1) at (0,0) {1};
\node[main node] (v2) at (3,0) {2};
\node[main node] (v3) at (2.6,1) {3};
\node[main node] (v4) at (-.4,2) {4};
\node[main node] (v5) at (1.5,3) {5};
\end{scope}
\begin{scope}[every edge/.style={draw=black,ultra thick}]
\draw  (v1) edge node{$\sfrac{1}{2}$} (v2);
\draw  (v1) edge node{$\sfrac{1}{4}$} (v3);
\draw  (v1) edge node{$\sfrac{1}{4}$} (v4);
\draw  (v2) edge node{$\sfrac{1}{4}$} (v3);
\draw  (v3) edge node{$\sfrac{1}{4}$} (v4);
\draw  (v3) edge node{$\sfrac{1}{4}$} (v5);
\draw  (v4) edge node{$\sfrac{1}{4}$} (v5);
\end{scope}
\end{tikzpicture}
\end{figure}

$\beta ^\star$ $\rightarrow$ maksimum funkcji $\sum _{e\in E(G)} x_e\ t$ że $\sum _{e:v\in e} x_e\leq 1$ dla każdego $v \in V (G)$.
\begin{align*}
&\begin{matrix}
1: &\frac{1}{2}+\frac{1}{4}+\frac{1}{4}=1\\
2: &\frac{1}{2}+\frac{1}{4}=\frac{3}{4}\\
3: &\frac{1}{4}+\frac{1}{4}+\frac{1}{4}=\frac{3}{4}\\
4: &\frac{1}{4}+\frac{1}{4}+\frac{1}{4}=\frac{3}{4}\\
5: &\frac{1}{4}+\frac{1}{4}=\frac{1}{2}
\end{matrix}
\end{align*}
$$\beta ^\star =\frac{1}{4}+\frac{1}{4}+\frac{1}{4}+\frac{1}{4}+\frac{1}{4}+\frac{1}{4}+\frac{1}{2}=2$$

\item $\gamma ^\star \leq 2,5$
\begin{figure}[H]
\centering
\begin{tikzpicture}[shorten >=1pt, auto, node distance=3cm, ultra thick,main node/.style={circle,draw,minimum size=.1cm,inner sep=0pt]}]%,fill=black
\begin{scope}%[every node/.style={font=\sffamily\Large\bfseries}]%[main node]
\node[main node] (v1) at (0,0)[label=below:$\sfrac{1}{2}$] {1};
\node[main node] (v2) at (3,0)[label=below:$\sfrac{1}{2}$] {2};
\node[main node] (v3) at (3,3)[label=above:$\sfrac{1}{2}$] {3};
\node[main node] (v4) at (0,3)[label=left:$\sfrac{3}{4}$] {4};
\node[main node] (v5) at (-2,4.5)[label=above:$\sfrac{1}{4}$] {5};
\end{scope}
\begin{scope}[every edge/.style={draw=black,ultra thick}]
\draw  (v1) edge (v2);
\draw  (v1) edge (v4);
\draw  (v2) edge (v3);
\draw  (v3) edge (v4);
\draw  (v4) edge (v5);
\end{scope}
\end{tikzpicture}
\end{figure}
$\gamma ^\star$ $\rightarrow$ minimum funkcji $\sum _{v\in V(G)} x_v\ t$ że $\sum _{v:v\in e} x_v \geq 1$ dla każdego $e \in E(G)$.
\begin{align*}
&\begin{matrix}
1\leftrightarrow 2: & \frac{1}{2}+\frac{1}{2}=1\\
1\leftrightarrow 4: & \frac{1}{2}+\frac{3}{4}=\frac{5}{4}\\
2\leftrightarrow 1: & \frac{1}{2}+\frac{1}{2}=1\\
2\leftrightarrow 3: & \frac{1}{2}+\frac{1}{2}=1\\
3\leftrightarrow 2: & \frac{1}{2}+\frac{1}{2}=1\\
3\leftrightarrow 4: & \frac{1}{2}+\frac{3}{4}=\frac{5}{4}\\
4\leftrightarrow 1: & \frac{3}{4}+\frac{1}{2}=\frac{5}{4}\\
4\leftrightarrow 3: & \frac{3}{4}+\frac{1}{2}=\frac{5}{4}\\
4\leftrightarrow 5: & \frac{3}{4}+\frac{1}{4}=1
\end{matrix}
&\gamma ^\star = \frac{1}{4}+\frac{3}{4}+\frac{1}{2}+\frac{1}{2}+\frac{1}{2}=2.5
\end{align*}



\item $\omega ^\star \geq 3$
\begin{figure}[H]
\centering
\begin{tikzpicture}[shorten >=1pt, auto, node distance=3cm, ultra thick,main node/.style={circle,draw,minimum size=.1cm,inner sep=0pt]}]%fill=black,
\begin{scope}%[every node/.style={font=\sffamily\Large\bfseries}]%[main node]
\node[main node] (v1) at (0,0)[label=below:$1$] {1};
\node[main node] (v2) at (3,0)[label=below:$\sfrac{1}{2}$] {2};
\node[main node] (v3) at (3,3)[label=above:$1$] {3};
\node[main node] (v4) at (0,3)[label=above:$\sfrac{1}{2}$] {4};
\end{scope}
\begin{scope}[every edge/.style={draw=black,ultra thick}]
\draw  (v1) edge (v2);
\draw  (v1) edge (v4);
\draw  (v1) edge (v3);
\draw  (v2) edge (v3);
\draw  (v3) edge (v4);
\end{scope}
\end{tikzpicture}
\end{figure}
$\omega ^\star$ $\rightarrow$ maksimum funkcji $\sum _{v\in V(G)} x_v$ takiej że $\sum _{v:v\in I} x_v \leq 1$ dla każdego zbioru niezależnego $I \subseteq V(G)$.
$$\{\underset{\sfrac{1}{2}}{\{4,2\}},\underset{1}{\{1\}},\underset{\sfrac{1}{2}}{\{2\}},\underset{1}{\{3\}},\underset{\sfrac{1}{2}}{\{4\}}\}$$
$$\omega ^\star = 3$$
$$\chi (G)\geq \chi ^\star (G)=\omega ^\star (G)\geq\omega (G)$$
$\chi (G) =3\ \ \omega (G) =3$
\end{enumerate}


\paragraph{A4}
\begin{itemize}
\item Czy dla grafu (a) z poprzedniego zadania zachodzi $\gamma ^\star \geq 2$ ?

\textbf{Odpowiedz: }Wynika z twierdzenia o dualnoci: 
$$\gamma (G) \geq \gamma ^\star (G)=\beta ^\star (G)\geq \beta (G)$$
Więc tak zachodzi
\item Czy dla grafu (b) z poprzedniego zadania zachodzi $\beta ^\star > 2,5$ ?

\textbf{Odpowiedź: }Jak wyżej - nie bangla
\item Czy dla grafu (c) z poprzedniego zadania zachodzi $\chi ^\star = 3$ ?

\textbf{Odpowiedź: }
$$\chi (G)\geq \chi ^\star(G)=\omega ^\star(G)\geq \omega (G)$$
więc tak.
\end{itemize}


\paragraph{A5} Niech $G$ będzie grafem (c) z zadania A3.
\begin{enumerate}[label=\alph*)]
\item Wypisz wszystkie pięć zbiorów niezależnych grafu $G$.

\textbf{Odpowiedź:}
$$\{\{2,4\},\{1\},\{2\},\{3\},\{4\}\}$$
\item Każdemu zbioru niezależnemu $I \subseteq V(G)$ przyporządkuj liczbę nieujemną $y_I$ w taki sposób, by dla każdego wierzchołka w zachodził warunek $\sum _{I:w \in I}y_I \geq 1$ (sumujemy po zbiorach niezależnych).
$$\{\underset{0.5}{\{2,4\}},\underset{1}{\{1\}},\underset{0.5}{\{2\}},\underset{1}{\{3\}},\underset{0.5}{\{4\}}\}$$
\item Jakie z tego wynika oszacowanie na $\chi ^\star$?

$$\chi ^\star = \sum y_I=3.5$$
\item Jakie z tego wynika oszacowanie na $\omega ^\star$?

Z twierdzenia o dualności wynika, że $\omega ^\star (G) = \chi ^\star (G)$
\end{enumerate}

\paragraph{A6} Dla podanych grafów:\\
\begin{minipage}{.5\textwidth}
a)
\begin{tikzpicture}[shorten >=1pt, auto, node distance=3cm, ultra thick,main node/.style={circle,fill=black,draw,minimum size=.1cm,inner sep=0pt]}]
\node[main node] (v1) at (0,0) {};
\node[main node] (v2) at (-1.5,1.5) {};
\node[main node] (v3) at (1.5,1.5) {};
\node[main node] (v5) at (-2.5,3) {};
\node[main node] (v4) at (2.5,3) {};
\draw  (v1) edge (v2);
\draw  (v2) edge (v3);
\draw  (v3) edge (v1);
\draw  (v3) edge (v4);
\draw  (v5) edge (v2);
\end{tikzpicture}
\end{minipage}%
\begin{minipage}{.5\textwidth}
b)
\begin{tikzpicture}[shorten >=1pt, auto, node distance=3cm, ultra thick,main node/.style={circle,fill=black,draw,minimum size=.1cm,inner sep=0pt]}]
\node[main node] (v1) at (0,0) {};
\node[main node] (v2) at (2,1) {};
\node[main node] (v3) at (2,3) {};
\node[main node] (v4) at (0.5,4) {};
\node[main node] (v5) at (-1.5,3.5) {};
\node[main node] (v6) at (-2,2) {};
\node[main node] (v7) at (-1.5,0.5) {};
\draw  (v1) edge (v2);
\draw  (v2) edge (v3);
\draw  (v3) edge (v4);
\draw  (v4) edge (v5);
\draw  (v5) edge (v6);
\draw  (v6) edge (v7);
\draw  (v7) edge (v1);
\end{tikzpicture}
\end{minipage}
\begin{itemize}
\item Wyznacz liczbę chromatyczną $\chi$, liczbę klikową $\omega$, przykładowe najmniejsze pokrycie wierzchołkowe, i przykładowe największe skojarzenie.
\item Oszacuj jak najlepiej potrafisz z góry i z dołu parametry $\beta ^\star, \gamma ^\star, \omega ^\star, \chi ^\star$.
\end{itemize}
\begin{enumerate}[label=\alph*)]
\item $\chi = 3\ \ \omega = 3\ \ \gamma = 2\ \ \beta = 2$
\item $\chi = 3\ \ \omega = 2\ \ \gamma = 4\ \ \beta = 3$
\end{enumerate}
\begin{enumerate}[label=\alph*)]
\item $\chi ^\star\leq 3\ \ \omega ^\star\geq 3\ \ \gamma ^\star\leq 2\ \ \beta ^\star\geq 2$
\item $\chi ^\star\leq 3\ \ \omega ^\star\geq 2\ \ \gamma ^\star\leq 4\ \ \beta ^\star\geq 3$
\end{enumerate}

\paragraph{A7} Oceń poprawność każdego z poniższych zdań. W każdym przypadku poprzyj odpowiedź, w zależności od potrzeby, uzasadnieniem ogólnym, przykładem lub kontrprzykładem.
\begin{enumerate}[label=\alph*)]
\item Dla każdego grafu $G$ moc największego skojarzenia jest nie większa niż $\beta ^\star(G)$.

$$\forall _{G} \beta (G) \not > \beta ^\star (G)$$
\textbf{Odpowiedź: }Z faktu na wykładzie wiemy, że $\beta ^\star (G)\geq \beta (G)$ więc, jest to \textbf{PRAWDA}
\item Dla każdego grafu $G$ moc najmniejszego pokrycia wierzchołkowego jest nie mniejsza niż $\gamma ^\star(G)$.

$$\forall _G \gamma (G) \not < \gamma ^\star(G)$$
\textbf{Odpowiedź: }Z faktu podanego, że $\gamma (G)\geq \gamma ^\star(G)$ wynika, że jest to \textbf{PRAWDA}
\item W każdym grafie ułamkowa liczba skojarzenia jest równa ułamkowej liczbie pokryciowej.

$$\forall _G \beta ^\star (G)=\gamma ^\star (G)$$
\textbf{Odpowiedź: }Z faktu podanego na wykładzie, że $\gamma (G) \geq \gamma ^\star (G)=\beta ^\star (G)\geq \beta (G)$ wynika, że podaje zdanie jest \textbf{PRAWDZIWE}
\item W każdym grafie moc największego skojarzenia jest równa mocy najmniejszego pokrycia.

$$\forall _G \beta (G)=\gamma (G)$$
\textbf{Odpowiedź: }Z faktu podanego na wykładzie, że $\gamma (G) \geq \gamma ^\star (G)=\beta ^\star (G)\geq \beta (G)$ wynika, że podaje zdanie jest \textbf{NIEPRAWDZIWE} 
\begin{figure}[H]
\centering
\begin{tikzpicture}[shorten >=1pt, auto, node distance=3cm, ultra thick,main node/.style={circle,fill=black,draw,minimum size=.1cm,inner sep=0pt]}]
\node[main node] (v1) at (0,0) {};
\node[main node] (v2) at (2,0) {};
\node[main node] (v3) at (1,1) {};
\draw  (v1) edge (v2);
\draw  (v2) edge (v3);
\draw  (v3) edge (v1);
\end{tikzpicture}
\caption*{$\beta = 1\ \gamma =2$}
\end{figure}
\item W każdym grafie ułamkowa liczba klikowa jest równa ułamkowej liczbie chromatycznej.

$$\forall _G\omega ^\star (G)=\chi ^\star (G)$$
\textbf{Odpowiedź: }Z faktu podanego na wykładzie, że $\chi (G)\geq \chi ^\star(G)=\omega ^\star(G)\geq \omega (G)$ wynika, że podane zdanie jest \textbf{PRAWDZIWE}.
\item W każdym grafie moc największej kliki jest równa liczbie chromatycznej.

$$\forall _G \omega (G)=\chi (G)$$
\textbf{Odpowiedź: }Z faktu podanego na wykładzie, że $\chi (G)\geq \chi ^\star(G)=\omega ^\star(G)\geq \omega (G)$ wynika, że podane zdanie \textbf{NIE JEST PRAWDZIWE}
\begin{figure}[H]
\centering
\begin{tikzpicture}[shorten >=1pt, auto, node distance=3cm, ultra thick,main node/.style={circle,fill=black,draw,minimum size=.1cm,inner sep=0pt]}]
\node[main node] (v1) at (0,0) {};
\node[main node] (v2) at (1,0) {};
\node[main node] (v3) at (1,1) {};
\node[main node] (v4) at (0.5,1.5) {};
\node[main node] (v5) at (0,1) {};
\draw  (v1) edge (v2);
\draw  (v2) edge (v3);
\draw  (v3) edge (v4);
\draw  (v4) edge (v5);
\draw  (v5) edge (v1);
\end{tikzpicture}
\caption*{$\omega = 2\ \chi =3$}
\end{figure}
\end{enumerate}

\subsection{Zadania domowe B}
\paragraph{B1} Przypuśćmy, że chcemy Zminimalizować funkcję
$f(x, y, z, w, u) = w - u$,
przy ograniczeniach zadanych nierównościami obok. Użyj dualności i sprowadź ten problem do równoważnego mu problemu maksymalizacji odpowiedniej funkcji. Nie rozwiązuj problemu.
$$\left\{\begin{matrix}
-x &- 2y &- z &- 3w &+ u &\geq 0\\
-2x &- y &- 2z &&+ u&\geq 0\\
&-2y &- z &- 2w &+ u &\geq 0\\
-x &&- 2z &- w &+ u &\geq 0\\
-2x &- 2y &- 2z &- 2w &+ u &\geq 0\\
x &+ y &+ z &+ w &&\geq 1 \end{matrix}\right.$$


\paragraph{B2} Dobra wróżka powiedziała, że podana funkcja $f_1$ na obszarze ograniczonym podanymi nierównościami osiąga minimum dla $(x; y; z; u) = (0; 0; 0,29; 0,22)$. Czy na tej podstawie potrafisz wyznaczyć, jaka jest największą wartość funkcji $f_2$ przy podanym poniżej niej ograniczeniach?

\begin{minipage}{.5\textwidth}
$$\left\{\begin{matrix}
f_1(x, y, z, u) = 16x + 8y + 8z\\
4x - 8y + 9u \geq 2\\
6x + 4y + 7z - 9u \geq 0\\
y + 5z + 2u \geq -5\\
\end{matrix}\right.$$
\end{minipage}%
\begin{minipage}{.5\textwidth}
$$\left\{\begin{matrix}
f_2(x, y, z) = 2x - 5z\\
4x + 6y \leq 16\\
-8x + 4y + z \leq 8\\
7y + 5z \leq 8\\
9x - 9y + 2z \leq 0\\
\end{matrix}\right.$$
\end{minipage}



\paragraph{B3} Przypuśćmy, że chcemy Zmaksymalizować funkcję
$f(x, y) = x - 3y$ przy podanych obok ograniczeniach.
\begin{enumerate}[label=\alph*)]
\item Sformułuj zagadnienie dualne do powyższego.
\item Wyznacz szukane maksimum funkcji $f$ i minimum funkcji z zagadnienia dualnego.
\end{enumerate}
$$\left\{\begin{matrix}
x &- y &\leq -1\\
x &&\leq 2\\
-2x&+ y &\leq 3\\
\end{matrix}\right.$$

\paragraph{B4} Oszacuj jak najlepiej potrafisz $\chi ^\star(G)$, z góry i z dołu, gdy $G$ jest cyklem na trzynastu wierzchołkach.

\paragraph{B5} Oszacuj jak najlepiej potrafisz $\beta ^\star, \gamma ^\star \text{ i } \chi ^\star(G)$, z góry i z dołu, gdy
\begin{enumerate}[label=\alph*)]
\item  $G$ jest grafem powstałym z cyklu na trzynastu wierzchołkach, przez dodanie do niego krawędzi łączących wierzchołki posiadające wspólnego sąsiada (czyli dodajemy 13 krawędzi).
\item  $G = K_{3,5}$.
\end{enumerate}


\subsection{Zadania na ćwiczeniach}
\paragraph{Zad.1} Rozwiąż podane zagadnienie programowania liniowego i zagadnienie do niego dualne.
$$\left\{\begin{matrix}
f(x_1, x_2, x_3) = x_1 + 2x_2 (max)\\
x_1 + x_2 + x_3 \leq 2\\
2x_2 - x_3 \leq 1
\end{matrix}\right.$$

\paragraph{Zad.2}
\begin{enumerate}[label=\alph*)]
\item Oblicz $\beta ^\star (G)$  i $\gamma ^\star (G)$, dla podanych grafów.
\item Podaj rozwiązania dla odpowiadających tym parametrom zagadnień programowania liniowego (jakie liczby wierzchołkom/krawędziom przyporządkować?)
\end{enumerate}


\paragraph{Zad.3}
\begin{enumerate}[label=\alph*]
\item Oblicz $\omega ^\star (G)$ i $\chi ^\star (G)$, dla podanych grafów.
\item Podaj rozwiązania dla odpowiadających tym parametrom zagadnień programowania liniowego (jakie liczby wierzchołkom/zbiorom niezależnym przyporządkować?)
\end{enumerate}

\paragraph{Zad.4}
\begin{enumerate}
\item Sformułuj zagadnienie programowania całkowitoliczbowego, znajdujące liczbę niezależności $\alpha (G)$ grafu.
\item Sformułuj zagadnienie programowania liniowego, znajdujące ułamkową liczbę niezależności $\alpha ^\star(G)$ grafu.
\item Sformułuj zagadnienie dualne do poprzedniego. Z jakiem parametrem grafowym jest związane?
\end{enumerate}



%-------------------- WYKŁAD --------------------
\end{document}
