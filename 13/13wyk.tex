\documentclass[a4paper,12pt]{article}

%% Language and font encodings
\usepackage[T1]{fontenc}
\usepackage[polish]{babel}
\usepackage[utf8]{inputenc}
\usepackage{lmodern}
\selectlanguage{polish}

%% Sets page size and margins
%\usepackage[a4paper,top=2cm,bottom=2cm,left=2cm,right=4cm,asymmetric]{geometry}
\usepackage{geometry}


%% Useful packages
\usepackage[fleqn]{amsmath}%[fleqn]
\usepackage{xfrac} %\sfrac{}{}

\usepackage{titlesec}%titles
\titlelabel{\thetitle.\quad}
\let\savenumberline\numberline
\def\numberline#1{\savenumberline{#1.}}
\usepackage{etoolbox}%dots in TOC
\makeatletter
\patchcmd{\l@section}
{\hfil}
{\leaders\hbox{\normalfont$\m@th\mkern \@dotsep mu\hbox{.}\mkern \@dotsep mu$}\hfill}
{}{}
\makeatother

\usepackage{caption,graphicx}
\usepackage{float}
\usepackage{sidecap}
\usepackage[colorinlistoftodos]{todonotes}
\usepackage[colorlinks=true, allcolors=blue]{hyperref}

\usepackage{tikz}
\usepackage{tikz-qtree}
\usetikzlibrary{trees}
\usetikzlibrary{arrows,positioning,shapes,fit,calc,decorations.pathreplacing}
\usetikzlibrary{graphs}
\usetikzlibrary{graphs.standard}
\usepackage{forest}
\usepackage{tikzscale}
\usepackage{pgfgantt} %Diagramy gantta http://bay.uchicago.edu/CTAN/graphics/pgf/contrib/pgfgantt/pgfgantt.pdf
\usepackage{pgf}
\usepackage{caption}

\usepackage{fancybox}
\usepackage{listings}

\usepackage[colorinlistoftodos]{todonotes} %http://mirror.unl.edu/ctan/macros/latex/contrib/todonotes/todonotes.pdf

\usepackage{array,longtable}

\newcommand\floor[1]{\lfloor#1\rfloor} %PODŁOGA -> \floor
\newcommand\ceil[1]{\lceil#1\rceil} %SUFIT -> \ceil

\usepackage{fancyhdr}
\pagestyle{fancy}
\rhead{\thepage}
\lhead{\leftmark}
\rfoot{\thepage}
\lfoot{\rightmark}

%http://tex.stackexchange.com/questions/64170/which-package-to-use-for-writing-algorithms
\usepackage{algorithm}% http://ctan.org/pkg/algorithms
\usepackage{algpseudocode}% http://ctan.org/pkg/algorithmicx
\newcommand{\var}[1]{{\ttfamily#1}}% variable

\algnewcommand\algorithmicforeach{\textbf{for each}} %Algorithm foreach
\algdef{S}[FOR]{ForEach}[1]{\algorithmicforeach\ #1\ \algorithmicdo}

\usepackage{amsthm}
\usepackage[inline]{enumitem} %enumerations
\usepackage{multicol}

\theoremstyle{definition}%~ %%% <-  Note that space!
\newtheorem{lemma}{Lemat} %\begin{lemma} ... \end{lemma} LEMAT(?)
\newtheorem{remark}{Wniosek}%\begin{remark} ... \end{remark} WNIOSEK 
%\newtheorem*{remark*}{Wniosek}%\begin{remark} ... \end{remark} WNIOSEK bez liczby
\newtheorem{theorem}{Twierdzenie}%\begin{theorem} ... \end{theorem}
\newtheorem{fact}{Fakt} %\begin{fact} ... \end{fact}
\newtheorem*{fact*}{Fakt} %\begin{fact*} ... \end{fact*} Fakt
\newtheorem*{observation*}{Obserwacja}

\newtheorem{example}{Przykład}
\newtheorem*{example*}{Przykład} %\begin{example} ... \end{example}
\theoremstyle{definition}
\newtheorem{definition}{Definicja}%\begin{definition}{} ... \end{definition}
%\newtheorem*{definition*}{Definicja}
%\newtheorem*{hipoterm*}{Hipoteza}%\begin{hipoterm*}[] ... \end{hipoterm*}
\newtheorem{hipoterm}{Hipoteza}%\begin{hipoterm*}[] ... \end{hipoterm*}
\theoremstyle{problem}
\newtheorem{problem}{Problem}%\begin{problem}{} ... \end{problem}
\newtheorem*{problem*}{Problem}

\let\originalforall=\forall%FORALL
\renewcommand{\forall}{\mathop{\vcenter{\hbox{\Large$\originalforall$}}}}
\let\originalexists=\exists%EXISTS
\renewcommand{\exists}{\mathop{\vcenter{\hbox{\Large$\originalexists$}}}}

\usepackage{cancel} %skreślenie równania \xcancel{...} \cancel{} lub \bcancel{}

\usepackage{amsfonts}

\usepackage{comment}

\usepackage{xcolor,colortbl}
\usepackage{multirow}

%\usepackage{wrapfig} %wrap text around figure

\usepackage{pdfpages}%\includepdf{file}

\usepackage{etoolbox}
\let\bbordermatrix\bordermatrix
\patchcmd{\bbordermatrix}{8.75}{4.75}{}{}
\patchcmd{\bbordermatrix}{\left(}{\left[}{}{}
\patchcmd{\bbordermatrix}{\right)}{\right]}{}{}
%\bbordermatrix{}

\allowdisplaybreaks

\title{Struktury Dyskretne - Notatki}
\author{Piotr Parysek\\
\href{mailto:piotr.parysek@outlook.com}{piotr.parysek@outlook.com} }
\date{\today}

\begin{document}
\maketitle

\tableofcontents
\section[Wykład 13: 8-VI-2017 - Temat: Ekspandery]{Temat: Ekspandery}
\begin{definition}[Ekspander]
Ekspander to graf w którym najmniejsza niezerowa wartość własna Laplasjanu leży daleko od zera.\\
EKSPANDER $\rightarrow$ graf o równomiernie rozłożonych krawędziach
\end{definition}
\begin{figure}[H]
\centering
\begin{tikzpicture}
\coordinate (A) at (0.45,0);
\coordinate (B) at (1.5,0);
\coordinate (D) at (3.5,0);
\coordinate (E) at (6.5,0);

\draw[->] (0,0) -- (7,0);

\draw[thick] ($(A)+(0,5pt)$) node[above] {$0$} -- ($(A)-(0,5pt)$);
\draw[thick] ($(B)+(0,5pt)$) node[above] {$\lambda _1$} -- ($(B)-(0,5pt)$);
\draw[thick] ($(D)+(0,5pt)$) node[above] {Wartości własne} -- ($(D)-(0,5pt)$);
\draw[thick] ($(E)+(0,5pt)$) node[above] {$2d$} -- ($(E)-(0,5pt)$);
\draw[decorate,decoration={brace,amplitude=5pt,mirror,raise=3mm}] (A) -- (B) node [black,midway,yshift=-7mm] {\footnotesize $\epsilon d$};
\end{tikzpicture}
\end{figure}

\subsection{Zastosowanie ekspanderów}
\subsubsection{Teoria kodów}
\paragraph{Kody Spiser - Spielman} wymyślone przez: M. Spiser D. Spielman
\begin{figure}[H]
\centering
\begin{tikzpicture}
\tikzstyle{arrow}=[->, draw, thick]
\node[draw, circle] (v3) at (0,0) {};
\node[draw, circle] (v6) at (-4.5,-2.5) {};
\node[draw, circle] (v10) at (-3,-2.5) {};
\node[draw, circle] (v7) at (-1.5,-2.5) {};
\node[draw, circle] (v11) at (0,-2.5) {};
\node[draw, circle] (v8) at (1.5,-2.5) {};
\node[draw, circle] (v12) at (3,-2.5) {};
\node[draw, circle] (v9) at (4.5,-2.5) {};
\node[draw, circle] (v13) at (6,-2.5) {};
\node[draw, circle] (v2) at (-1.5,0) {};
\node[draw, circle] (v4) at (1.5,0) {};
\node[draw, circle] (v1) at (-3,0) {};
\node[draw, circle] (v5) at (3,0) {};
\node (v0) at (-4,0) {\textbf{U:}};
\node (v00) at (-5.5,-2.5) {\textbf{W:}};

\draw[arrow]  (v1) edge (v6);
\draw[arrow]  (v1) edge (v7);
\draw[arrow]  (v1) edge (v8);
\draw[arrow]  (v1) edge (v9);
\draw[arrow]  (v2) edge (v10);
\draw[arrow]  (v2) edge (v11);
\draw[arrow]  (v2) edge (v12);
\draw[arrow]  (v2) edge (v13);
\draw[arrow]  (v3) edge (v6);
\draw[arrow]  (v3) edge (v11);
\draw[arrow]  (v3) edge (v9);
\draw[arrow]  (v3) edge (v10);
\draw[arrow]  (v4) edge (v7);
\draw[arrow]  (v4) edge (v12);
\draw[arrow]  (v4) edge (v13);
\draw[arrow]  (v4) edge (v8);
\draw[arrow]  (v5) edge (v6);
\draw[arrow]  (v5) edge (v13);
\draw[arrow]  (v5) edge (v11);
\draw[arrow]  (v5) edge (v8);
\end{tikzpicture}
\end{figure}
$$\bbordermatrix{
&\Huge U &\Huge W\cr
\Huge U& \Huge O &\Huge H\cr
\Huge W & \Huge H^T & \Huge O 
}$$
Przypuśćmy, że graf $G$ jest ekspanderem.\\
Wtedy bardzo szybko można skorygować 2 błędy.
\begin{description}
\item[$U_1$] $|U|=100$, każdy wierzchołek $U$ jest połączony z 20 wierzchołkami z $W$
\item[$W_1$] $|W|=1000$
\end{description}
Taki graf (jeśli jest ekspanderem) potrafi skorygować nawet $5\%$ błędów.

\subsubsection{Liczenie średniej odległości}
\begin{theorem}[Metatwerdzenie]
Ekspander to rzadki graf który często ,,zachowuje się'' jak pełny.
\end{theorem}
\begin{proof}
Przypuśćmy, że mamy $n=10^5$ punktów, który jest ekspanderem.\\
Średnia odległość liczona po krawędziach tego grafu przybliża bardzo dobrze średnią odległość w liczonym grafie.
\end{proof}

\subsubsection{,,Zmniejszenie prawdopodobieństwa''}
\begin{theorem}[Małe twierdzenie Fermata]
$$a^{p-1}\equiv 1 (\mod p)$$
\end{theorem}
Przypuśćmy, że $n$ jest liczbą złożoną
$$a^{n-1}\equiv 1(\mod m)$$
dla co najwyżej potęgi liczb całkowitej produktu $[1;n-1]$
\begin{figure}[H]
\centering 
\begin{tikzpicture}
\draw  (0,0) ellipse (1 and 2);
\draw  (3,0) ellipse (.5 and 1);
\draw[ultra thick,color=green] (.8,1) arc (0:180:.8cm);
\draw[ultra thick,color=green] (-.8,1) -- (.8,1);
\node[color=green] at (0,3) {fałszywi świadkowie};
\draw[->,ultra thick,color=green] (0,2.8) -- (0,1.9);
\draw[ultra thick,color=blue]  (3,.65) ellipse (.25 and .25);
\draw[ultra thick,color=blue] (.55,1.2) -- (3,.65);
\node[ultra thick,color=blue] at (4,2) {$\leq \frac{n}{2*9}=\frac{n}{18}$};
\draw[->,ultra thick,color=blue] (4,1.7) -- (2.85,.8);
\end{tikzpicture}
\end{figure}
$d=10$ $|S|\leq \frac{n}{19}$ $|N(S)|> 9|S|$

I tak zamiast $\binom{1}{2}^k$ uzyskujemy $\binom{1}{19}^k$

%----------------------------------------------------------------

\end{document}
