\documentclass[a4paper,12pt]{article}

%% Language and font encodings
\usepackage[T1]{fontenc}
\usepackage[polish]{babel}
\usepackage[utf8]{inputenc}
\usepackage{lmodern}
\selectlanguage{polish}

%% Sets page size and margins
%\usepackage[a4paper,top=2cm,bottom=2cm,left=2cm,right=4cm,asymmetric]{geometry}
\usepackage{geometry}


%% Useful packages
\usepackage[fleqn]{amsmath}%[fleqn]
\usepackage{xfrac} %\sfrac{}{}

\usepackage{titlesec}%titles
\titlelabel{\thetitle.\quad}
\let\savenumberline\numberline
\def\numberline#1{\savenumberline{#1.}}
\usepackage{etoolbox}%dots in TOC
\makeatletter
\patchcmd{\l@section}
{\hfil}
{\leaders\hbox{\normalfont$\m@th\mkern \@dotsep mu\hbox{.}\mkern \@dotsep mu$}\hfill}
{}{}
\makeatother

\usepackage{caption,graphicx}
\usepackage{float}
\usepackage{sidecap}
\usepackage[colorinlistoftodos]{todonotes}
\usepackage[colorlinks=true, allcolors=blue]{hyperref}

\usepackage{tikz}
\usepackage{tikz-qtree}
\usetikzlibrary{trees}
\usetikzlibrary{arrows,positioning,shapes,fit,calc,decorations.pathreplacing}
\usetikzlibrary{graphs}
\usetikzlibrary{graphs.standard}
\usepackage{forest}
\usepackage{tikzscale}
\usepackage{pgfgantt} %Diagramy gantta http://bay.uchicago.edu/CTAN/graphics/pgf/contrib/pgfgantt/pgfgantt.pdf
\usepackage{pgf}
\usepackage{caption}

\usepackage{fancybox}
\usepackage{listings}

\usepackage[colorinlistoftodos]{todonotes} %http://mirror.unl.edu/ctan/macros/latex/contrib/todonotes/todonotes.pdf

\usepackage{array,longtable}

\newcommand\floor[1]{\lfloor#1\rfloor} %PODŁOGA -> \floor
\newcommand\ceil[1]{\lceil#1\rceil} %SUFIT -> \ceil

\usepackage{fancyhdr}
\pagestyle{fancy}
\rhead{\thepage}
\lhead{\leftmark}
\rfoot{\thepage}
\lfoot{\rightmark}

%http://tex.stackexchange.com/questions/64170/which-package-to-use-for-writing-algorithms
\usepackage{algorithm}% http://ctan.org/pkg/algorithms
\usepackage{algpseudocode}% http://ctan.org/pkg/algorithmicx
\newcommand{\var}[1]{{\ttfamily#1}}% variable

\algnewcommand\algorithmicforeach{\textbf{for each}} %Algorithm foreach
\algdef{S}[FOR]{ForEach}[1]{\algorithmicforeach\ #1\ \algorithmicdo}

\usepackage{amsthm}
\usepackage[inline]{enumitem} %enumerations
\usepackage{multicol}

\theoremstyle{definition}%~ %%% <-  Note that space!
\newtheorem{lemma}{Lemat} %\begin{lemma} ... \end{lemma} LEMAT(?)
\newtheorem{remark}{Wniosek}%\begin{remark} ... \end{remark} WNIOSEK 
%\newtheorem*{remark*}{Wniosek}%\begin{remark} ... \end{remark} WNIOSEK bez liczby
\newtheorem{theorem}{Twierdzenie}%\begin{theorem} ... \end{theorem}
\newtheorem{fact}{Fakt} %\begin{fact} ... \end{fact}
\newtheorem*{fact*}{Fakt} %\begin{fact*} ... \end{fact*} Fakt
\newtheorem*{observation*}{Obserwacja}

\newtheorem{example}{Przykład}
\newtheorem*{example*}{Przykład} %\begin{example} ... \end{example}
\theoremstyle{definition}
\newtheorem{definition}{Definicja}%\begin{definition}{} ... \end{definition}
%\newtheorem*{definition*}{Definicja}
%\newtheorem*{hipoterm*}{Hipoteza}%\begin{hipoterm*}[] ... \end{hipoterm*}
\newtheorem{hipoterm}{Hipoteza}%\begin{hipoterm*}[] ... \end{hipoterm*}
\theoremstyle{problem}
\newtheorem{problem}{Problem}%\begin{problem}{} ... \end{problem}
\newtheorem*{problem*}{Problem}

\let\originalforall=\forall%FORALL
\renewcommand{\forall}{\mathop{\vcenter{\hbox{\Large$\originalforall$}}}}
\let\originalexists=\exists%EXISTS
\renewcommand{\exists}{\mathop{\vcenter{\hbox{\Large$\originalexists$}}}}

\usepackage{cancel} %skreślenie równania \xcancel{...} \cancel{} lub \bcancel{}

\usepackage{amsfonts}

\usepackage{comment}

\usepackage{xcolor,colortbl}
\usepackage{multirow}

%\usepackage{wrapfig} %wrap text around figure

\usepackage{pdfpages}%\includepdf{file}

\usepackage{etoolbox}
\let\bbordermatrix\bordermatrix
\patchcmd{\bbordermatrix}{8.75}{4.75}{}{}
\patchcmd{\bbordermatrix}{\left(}{\left[}{}{}
\patchcmd{\bbordermatrix}{\right)}{\right]}{}{}
%\bbordermatrix{}

\allowdisplaybreaks

\title{Struktury Dyskretne - Notatki}
\author{Piotr Parysek\\
\href{mailto:piotr.parysek@outlook.com}{piotr.parysek@outlook.com} }
\date{\today}

\begin{document}
\maketitle

\tableofcontents
\section[Wykład 6: 30-III-2017 - Temat: Przepływy w sieciach]{Temat: Przepływy w sieciach}
\subsection{Sieć}
\begin{definition}[Sieć]
Siecią nazywamy piątkę: $$S=(V,A,s,t,c)$$ gdzie
\begin{itemize}
\item[$V,A$] $\rightarrow $ graf skierowany o zbiorze wierzchołków $V$ i zbiorze strzałek $A$,
\item[$s$] $\rightarrow $ $s\in V$ wyróżniony wierzchołek nazywany \textbf{Źródłem},
\item[$t$] $\rightarrow $ $t\in V$ wyróżniony wierzchołek nazywany \textbf{Ujściem},
\item[$c$] $\rightarrow $ $c:A\rightarrow \mathbb{R} _+\cup \{0\}$ funkcja przyporządkowująca strzałkom wartość $\mathbb{R} _+\cup \{0\}$ nazywamy \textbf{funkcją pojemności}
\end{itemize}
\end{definition}
\begin{definition}[Graf skierowany]
Graf $D=(V,A)$ gdzie 
\begin{itemize}
\item[$V$] $\rightarrow $ zbiór wierzchołków,
\item[$A$] $\rightarrow $ $A\subseteq \{(v,w):v,w\in V\}$
\end{itemize}
\end{definition}
\begin{definition}[Przepływ sieci]
Przepływem sieci $S=(V,A,s,t,c)$ nazywamy funkcję $$f:A\rightarrow \mathbb{R}_+\cup \{0\}$$ spełniająca następujące warunki:
\begin{enumerate}[label=\alph*)]
\item $\forall _{a\in A}: 0\leq f(a)\leq c(a)$
\item $\forall _{v\in V,\, v\neq s,\, t}$
$$\sum _{\vec{wv}\in A}f(\vec{wv})=\sum _{\vec{vu}}f(\vec{vu})$$
\end{enumerate}
\end{definition}
\begin{definition}[Wartość przepływu]
Wartość przepływu $f$ jest zdefiniowana jako:
$$\mathsf{val}(f)=\sum _{\vec{sv}\in A}f(\vec{sv})=\sum _{\vec{ut}\in A} f(\vec{ut})$$
\end{definition}

\begin{algorithm*}
\caption{alg:Ford-Fulkerson}\label{alg:Ford-Fulkerson}
\begin{algorithmic}[1]
\Procedure{Ford-Fulkerson}{}
\State Zaczynamy od dowolnego przepływu $f$
\While{Jeśli jeszcze można}
\State Modyfikujemy przepływ zamieniając go na $f'$ żeby $$\mathsf{val}(f')>\mathsf{val}(f)$$
\State $f=f'$
\EndWhile
\EndProcedure
\end{algorithmic}
\end{algorithm*}

\begin{example*}[Znajdź przepływ n największej wartości $\mathsf{val}(f)$] Przykład zaprezentowany jako rysunki:\newline
\begin{minipage}{0.45\textwidth}
\begin{figure}[H]
\centering
\begin{tikzpicture}[shorten >=1pt, auto, node distance=3cm, ultra thick,main node/.style={circle,draw,minimum size=.4cm,inner sep=0pt]}]%fill=black,
\begin{scope}[every node/.style={font=\sffamily\Large\bfseries}]
\node[main node] (v1) at (0,0) {1};
\node[main node] (v2) at (2,0) {2};
\node[main node] (v3) at (4,2) {3};
\node[main node] (v4) at (4,-2) {4};
\node[main node] (v5) at (6,0) {5};
%\node[main node] (v) at (,) {};
\end{scope}
\begin{scope}[every edge/.style={draw=black,ultra thick}]
\draw[->]  (v1) edge node{$10$} (v2);
\draw[->]  (v2) edge node{$2$} (v3);
\draw[->]  (v2) edge node{$1$} (v4);
\draw[->]  (v3) edge node{$1$} (v5);
\draw[->]  (v4) edge node{$2$} (v5);
%\draw[->]  (v) edge node{} (v);
\end{scope}
\end{tikzpicture}
\caption*{Przykład: $s=1,\ t=3$}
\end{figure}
\end{minipage}

\begin{minipage}{0.95\textwidth}
\begin{figure}[H]
\centering
\begin{tikzpicture}[shorten >=1pt, auto, node distance=3cm, ultra thick,main node/.style={circle,draw,minimum size=.4cm,inner sep=0pt]}]%fill=black,
\begin{scope}[every node/.style={font=\sffamily\Large\bfseries}]
\node[main node] (v1) at (0,0) {1};
\node[main node] (v2) at (2,0) {2};
\node[main node] (v3) at (4,2) {3};
\node[main node] (v4) at (4,-2) {4};
\node[main node] (v5) at (6,0) {5};

%\node[main node] (v) at (,) {};
\end{scope}
\begin{scope}[every edge/.style={draw=black,ultra thick}]
\draw[->]  (v1) edge node{$\overset{2}{10}$} (v2);
\draw[->]  (v2) edge node{$\overset{1}{2}$} (v3);
\draw[->]  (v2) edge node{$\overset{1}{1}$} (v4);
\draw[->]  (v3) edge node{$\overset{1}{1}$} (v5);
\draw[->]  (v4) edge node{$\overset{3}{2}$} (v5);
%\draw[->]  (v) edge node{} (v);
\end{scope}
\end{tikzpicture}
\caption*{Przykład: $s=1,\ t=3$ z zaznaczonym przepływem, powyżej wartości wag}
\end{figure}
\end{minipage}

\begin{minipage}{0.95\textwidth}
\begin{figure}[H]
\centering
\begin{tikzpicture}[shorten >=1pt, auto, node distance=3cm, ultra thick,main node/.style={circle,draw,minimum size=.4cm,inner sep=0pt]}]%fill=black,
\begin{scope}[every node/.style={font=\sffamily\Large\bfseries}]
\node[main node] (v1) at (0,0) {1};
\node[main node] (v2) at (2,0) {2};
\node[main node] (v3) at (4,2) {3};
\node[main node] (v4) at (4,-2) {4};
\node[main node] (v5) at (6,0) {5};

%\node[main node] (v) at (,) {};
\end{scope}
\begin{scope}[every edge/.style={draw=black,ultra thick}]
\draw[->]  (v1) edge node{$\overset{2}{10}$} (v2);
\draw[->]  (v2) edge node{$\overset{0.5}{2}$} (v3);
\draw[->]  (v2) edge node{$\overset{1.5}{1}$} (v4);
\draw[->]  (v3) edge node{$\overset{0.5}{1}$} (v5);
\draw[->]  (v4) edge node{$\overset{1.5}{2}$} (v5);
%\draw[->]  (v) edge node{} (v);
\end{scope}
\end{tikzpicture}
\caption*{Przykład: $s=1,\ t=3$ z zaznaczonym przepływem, powyżej wartości wag}
\end{figure}
\end{minipage}


\begin{figure}[H]
\centering
\begin{tikzpicture}[shorten >=1pt, auto, node distance=3cm, ultra thick,main node/.style={circle,draw,minimum size=.4cm,inner sep=0pt]}]%fill=black,
\begin{scope}[every node/.style={font=\sffamily\Large\bfseries}]
\node[main node] (v1) at (0,0) {1};
\node[main node] (v2) at (2,2) {2};
\node[main node] (v3) at (4,2) {3};
\node[main node] (v4) at (6,2) {4};
\node[main node] (v5) at (2,0) {5};
\node[main node] (v6) at (4,0) {6};
\node[main node] (v7) at (6,0) {7};
\node[main node] (v8) at (2,-2) {8};
\node[main node] (v9) at (4,-2) {9};
\node[main node] (v10) at (6,-2) {10};
\node[main node] (v11) at (8,0) {11};
%\node[main node] (v) at (,) {};
\end{scope}
\begin{scope}[every edge/.style={draw=black,ultra thick}]
\draw[->]  (v1) edge node{11} (v2);
\draw[->]  (v1) edge node{3} (v5);
\draw[->]  (v1) edge node{5} (v8);
\draw[->]  (v2) edge node{11} (v3);
\draw[->]  (v2) edge node{2} (v5);
\draw[->]  (v3) edge node{10} (v4);
\draw[->]  (v3) edge node{1} (v6);
\draw[->]  (v3) edge node{2} (v5);
\draw[->]  (v4) edge node{6} (v7);
\draw[->]  (v4) edge node{1} (v11);
\draw[->]  (v5) edge node{8} (v6);
\draw[->]  (v6) edge node{8} (v7);
\draw[->]  (v7) edge node{2} (v3);
\draw[->]  (v7) edge node{4} (v11);
\draw[->]  (v7) edge node{7} (v10);
\draw[->]  (v8) edge node{1} (v5);
\draw[->]  (v8) edge node{2} (v6);
\draw[->]  (v8) edge node{2} (v9);
\draw[->]  (v9) edge node{2} (v6);
\draw[->]  (v9) edge node{1} (v10);
\draw[->]  (v10) edge node{2} (v6);
\draw[->]  (v10) edge node{10} (v11);
%\draw[->]  (v) edge node{} (v);
\end{scope}
\end{tikzpicture}
\caption*{$s=1,\ t=11$}
\end{figure}
\end{example*}

\begin{definition}[Ścieżka nienasycona] ciąg strzałek zaczynających się w $s$ a kończących się w $t$
\begin{figure}[H]
\centering
\begin{tikzpicture}[shorten >=1pt, auto, node distance=3cm, ultra thick,main node/.style={circle,draw,minimum size=.4cm,inner sep=0pt]}]%fill=black,
\begin{scope}[every node/.style={font=\sffamily\Large\bfseries}]
\node[main node] (v1) at (0,0) {$s$};
\node[main node] (v2) at (2,0) {};
\node[main node] (v3) at (4,0) {};
\node[main node] (v4) at (6,0) {};
\node[main node] (v5) at (8,0) {};
\node[main node] (v6) at (10,0) {};
\node[main node] (v7) at (12,0) {$t$};
%\node[main node] (v) at (,) {};
\end{scope}
\begin{scope}[every edge/.style={draw=black,ultra thick}]
\draw[->]  (v1) edge node{$\underset{2=2-0}{2}$} (v2);
\draw[->]  (v2) edge node{$\underset{3=5-2}{5}$} (v3);
\draw[->]  (v4) edge node{1} (v3);
\draw[->]  (v4) edge node{$\underset{2=7-5}{7}$} (v5);
\draw[->]  (v6) edge node{3} (v5);
\draw[->]  (v6) edge node{$\underset{1=2-1}{2}$} (v7);
%\draw[->]  (v) edge node{} (v);
\end{scope}
\end{tikzpicture}
\end{figure}
$$\min \{2,3,1,2,1,1\}=0$$
\end{definition}

\begin{definition}[Cięcie]
Cięciem w sieci $S=(V,A,s,t,c)$ nazywamy podział $V=S\cup \bar{S}$, takie że  $s\in S$, $t\not\in S$, $s\in \bar{S}$.

Pojemnością cięcia $(S,\bar{S})$ nazywamy sumą pojemności wszystkich strzałek z $S$ do $\bar{S}$ to: 
$$\mathsf{cap}(S,\bar{S})=\sum _{v\in S, w\in S}c(\vec{v,w})$$
\end{definition}


\begin{figure}[H]
\centering
\begin{tikzpicture}[shorten >=1pt, auto, node distance=3cm, ultra thick,main node/.style={circle,draw,minimum size=.4cm,inner sep=0pt]}]%fill=black,
\draw (2,2) ellipse (5cm and 2cm);
\draw[color=red] (4,4) -- (2,0);
\draw[->] (3,3) -> (5,3);
\draw[->] (3,3) -> (5,2); 
\draw[->] (2.5,2) -> (4.5,1.5); 
\draw[->] (1,2) -> (4.2,1); 
\draw (-2.5,2) circle (0.3) node {$s$};
\draw (6.5,2) circle (0.3) node {$t$};
\end{tikzpicture}
\end{figure}

\begin{fact*}
Dla dowolnego przepływu $f$ i dowolnego przecięcia $(S, \bar{S})$
$$\mathsf{val}(f)\leq \mathsf{cap}(S, \bar{S})$$
\end{fact*}

\begin{theorem}[MAX-FLOW-MIN-CUT] Dla każdej serii
$$\max _f \mathsf{val}(f)=\min _{(S, \bar{S})}\mathsf{cap}(S, \bar{S})$$
$$\mathsf{cap}(X,Y)\geq \mathsf{MINCUT}=\mathsf{MAXFLOW}\geq \mathsf{val}(f)$$
\end{theorem}

%--------------------------------------------------
\end{document}
