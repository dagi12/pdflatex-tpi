\documentclass[a4paper,12pt]{article}

%% Language and font encodings
\usepackage[T1]{fontenc}
\usepackage[polish]{babel}
\usepackage[utf8]{inputenc}
\usepackage{lmodern}
\selectlanguage{polish}

%% Sets page size and margins
%\usepackage[a4paper,top=2cm,bottom=2cm,left=2cm,right=4cm,asymmetric]{geometry}
\usepackage{geometry}


%% Useful packages
\usepackage[fleqn]{amsmath}%[fleqn]
\usepackage{xfrac} %\sfrac{}{}

\usepackage{titlesec}%titles
\titlelabel{\thetitle.\quad}
\let\savenumberline\numberline
\def\numberline#1{\savenumberline{#1.}}
\usepackage{etoolbox}%dots in TOC
\makeatletter
\patchcmd{\l@section}
{\hfil}
{\leaders\hbox{\normalfont$\m@th\mkern \@dotsep mu\hbox{.}\mkern \@dotsep mu$}\hfill}
{}{}
\makeatother

\usepackage{caption,graphicx}
\usepackage{float}
\usepackage{sidecap}
\usepackage[colorinlistoftodos]{todonotes}
\usepackage[colorlinks=true, allcolors=blue]{hyperref}

\usepackage{tikz}
\usepackage{tikz-qtree}
\usetikzlibrary{trees}
\usetikzlibrary{arrows,positioning,shapes,fit,calc,decorations.pathreplacing}
\usetikzlibrary{graphs}
\usetikzlibrary{graphs.standard}
\usepackage{forest}
\usepackage{tikzscale}
\usepackage{pgfgantt} %Diagramy gantta http://bay.uchicago.edu/CTAN/graphics/pgf/contrib/pgfgantt/pgfgantt.pdf
\usepackage{pgf}
\usepackage{caption}

\usepackage{fancybox}
\usepackage{listings}

\usepackage[colorinlistoftodos]{todonotes} %http://mirror.unl.edu/ctan/macros/latex/contrib/todonotes/todonotes.pdf

\usepackage{array,longtable}

\newcommand\floor[1]{\lfloor#1\rfloor} %PODŁOGA -> \floor
\newcommand\ceil[1]{\lceil#1\rceil} %SUFIT -> \ceil

\usepackage{fancyhdr}
\pagestyle{fancy}
\rhead{\thepage}
\lhead{\leftmark}
\rfoot{\thepage}
\lfoot{\rightmark}

%http://tex.stackexchange.com/questions/64170/which-package-to-use-for-writing-algorithms
\usepackage{algorithm}% http://ctan.org/pkg/algorithms
\usepackage{algpseudocode}% http://ctan.org/pkg/algorithmicx
\newcommand{\var}[1]{{\ttfamily#1}}% variable

\algnewcommand\algorithmicforeach{\textbf{for each}} %Algorithm foreach
\algdef{S}[FOR]{ForEach}[1]{\algorithmicforeach\ #1\ \algorithmicdo}

\usepackage{amsthm}
\usepackage[inline]{enumitem} %enumerations
\usepackage{multicol}

\theoremstyle{definition}%~ %%% <-  Note that space!
\newtheorem{lemma}{Lemat} %\begin{lemma} ... \end{lemma} LEMAT(?)
\newtheorem{remark}{Wniosek}%\begin{remark} ... \end{remark} WNIOSEK 
%\newtheorem*{remark*}{Wniosek}%\begin{remark} ... \end{remark} WNIOSEK bez liczby
\newtheorem{theorem}{Twierdzenie}%\begin{theorem} ... \end{theorem}
\newtheorem{fact}{Fakt} %\begin{fact} ... \end{fact}
\newtheorem*{fact*}{Fakt} %\begin{fact*} ... \end{fact*} Fakt
\newtheorem*{observation*}{Obserwacja}

\newtheorem{example}{Przykład}
\newtheorem*{example*}{Przykład} %\begin{example} ... \end{example}
\theoremstyle{definition}
\newtheorem{definition}{Definicja}%\begin{definition}{} ... \end{definition}
%\newtheorem*{definition*}{Definicja}
%\newtheorem*{hipoterm*}{Hipoteza}%\begin{hipoterm*}[] ... \end{hipoterm*}
\newtheorem{hipoterm}{Hipoteza}%\begin{hipoterm*}[] ... \end{hipoterm*}
\theoremstyle{problem}
\newtheorem{problem}{Problem}%\begin{problem}{} ... \end{problem}
\newtheorem*{problem*}{Problem}

\let\originalforall=\forall%FORALL
\renewcommand{\forall}{\mathop{\vcenter{\hbox{\Large$\originalforall$}}}}
\let\originalexists=\exists%EXISTS
\renewcommand{\exists}{\mathop{\vcenter{\hbox{\Large$\originalexists$}}}}

\usepackage{cancel} %skreślenie równania \xcancel{...} \cancel{} lub \bcancel{}

\usepackage{amsfonts}

\usepackage{comment}

\usepackage{xcolor,colortbl}
\usepackage{multirow}

%\usepackage{wrapfig} %wrap text around figure

\usepackage{pdfpages}%\includepdf{file}

\usepackage{etoolbox}
\let\bbordermatrix\bordermatrix
\patchcmd{\bbordermatrix}{8.75}{4.75}{}{}
\patchcmd{\bbordermatrix}{\left(}{\left[}{}{}
\patchcmd{\bbordermatrix}{\right)}{\right]}{}{}
%\bbordermatrix{}

\allowdisplaybreaks

\title{Struktury Dyskretne - Notatki}
\author{Piotr Parysek\\
\href{mailto:piotr.parysek@outlook.com}{piotr.parysek@outlook.com} }
\date{\today}

\begin{document}
\maketitle

\tableofcontents
\section[Wykład 12: 1-VI-2017 - Temat: Programowanie liniowe (i ułamkowe charakterystyki grafów)]{Temat: Programowanie liniowe (i ułamkowe charakterystyki grafów)}

\subsection*{Przypomnienie nazewnictwa}
\begin{description}
\item[klika] $\rightarrow$ graf pełny
\item[Liczba klikowa]  $\rightarrow$  największa klika w grafie
\begin{figure}[H]
\centering
\begin{tikzpicture}[shorten >=1pt, auto, node distance=3cm, ultra thick,main node/.style={circle,fill=black,draw,minimum size=.1cm,inner sep=0pt]}]%
\begin{scope}%[every node/.style={font=\sffamily\Large\bfseries}]%[main node]
\node[main node] (v1) at (0,0) {1};
\node[main node] (v2) at (2,0) {2};
\node[main node] (v3) at (1,1){3};
\node[main node] (v4) at (1,-1){4};
\node[main node] (v5) at (3,0) {5};
\node[main node] (v6) at (5,0) {6};
\node[main node] (v7) at (4,1) {7};
\node[main node] (v8) at (4,-1) {8};
\end{scope}
\begin{scope}[every edge/.style={draw=black,ultra thick}]
\draw  (v1) edge (v2);
\draw  (v1) edge (v3);
\draw  (v1) edge (v4);
\draw  (v2) edge (v3);
\draw  (v2) edge (v4);
\draw  (v2) edge (v5);
\draw  (v5) edge (v6);
\draw  (v5) edge (v7);
\draw  (v5) edge (v8);
\draw  (v6) edge (v7);
\draw  (v6) edge (v8);
\draw  (v7) edge (v8);
\end{scope}
\end{tikzpicture}
\caption*{Graf $G$ posiadający dwie kliki o wielkości 3 oraz klikę wielkości 4, czyli $\omega (G) =4$ }
\end{figure}
\end{description}



\subsection{Programowanie liniowe}
\begin{definition}[Silne twierdzenie dualności]
$$\max _{\bar{x}}f(\bar{x})=\min _{\bar{y}}g(\bar{y})$$
\end{definition}
\begin{problem*}[Znaleźć maksimum funkcji $f$]
\begin{align*}
&f(x_1,x_2,x_3)=6x_1+4x_2+x_3\\
&\left\{\begin{matrix}
3x_1 &+	2x_2 &+	x_3 &\leq 8\\
1x_1 &+	2x_2 &+	x_3 &\leq 5\\
1x_1 &+	0 &		x_3 &\leq 1\\
x_1, &	x_2, &	x_3	&\geq 0
\end{matrix}\right.\\
&\begin{matrix}
y_1:3x_1 &+	2x_2 &+	x_3 &\leq 8\\
y_2:x_1 &+	2x_2 &+ x_3 &\leq 5\\
y_3:x_1 &+		 &  x_3 &\leq 1
\end{matrix}\\
&\text{ZNALEĆ MINIMUM FUNKCJI FUNKCJI }g\\
&\left\{
\begin{matrix}
3y_1	&+ y_2	&+ y_3	&\geq 6\\
2y_1	&+ 2y_2	&& \geq 4\\
y_1		&+ y_2	&+ y_3	&\geq 1\\
y_1,	& y_2,	& y_3	&\geq 0
\end{matrix}
\right.\\
&\max _{x_1x_2x_3} f(x_1,x_2,x_3)\leq \min _{y_1y_2y_3} g(y_1,y_2,y_3)\\
&\text{ dla }y_1=0,\ y_2=2,\ y_3=4\\
&\min _{y_1y_2y_3} g(y_1,y_2,y_3)=14\\
&\text{ dla }x_1=1,\ x_2=2,\ x_3=0\\
&\max _{x_1x_2x_3} f(x_1,x_2,x_3)=14
\end{align*}
\end{problem*}

\begin{problem*}[Znaleźć maksimum funkcji $f$]
\begin{align*}
&f(\bar{x})=\bar{c}^T \bar{x}\\
&\text{przy ograniczeniu: }\\
&A\bar{x}\leq \bar{b}\\
&\bar{x} = \begin{bmatrix}
x_1\\x_2\\x_3
\end{bmatrix} & \bar{c}^T = \begin{bmatrix}
6&4&1
\end{bmatrix}\\
&A=\begin{bmatrix}
3&2&1\\1&2&1\\1&0&1
\end{bmatrix} & \bar{b}=\begin{bmatrix}
8\\5\\1
\end{bmatrix}
\end{align*}
Znaleźć minimum funkcji $g$
$$g(y)=b^Ty$$
przy ograniczeniach
\begin{align*}
&A^T\bar{y}\geq \bar{c}\\
&\bar{y} \geq 0
\end{align*}
\end{problem*}

\begin{theorem}
Jeśli $x^\star$ maksymalizuje po $f(x)$ a $y^\star$ minimalizuje po $g(y)$ wtedy:
$$f(x^\star)=c^Tx^\star =b^Ty^\star =g(y^\star)$$
\end{theorem}

\subsection{Ułamkowe charakterystyki grafów}
\begin{definition}[$\beta (G)$]
Niech $G$ będzie grafem o $n$ wierzchołkach i $m$ krawędziach.\\Niech $\beta (G)$ oznacza wielkość największego skojarzenia w grafie.
\end{definition}

\begin{figure}[H]
\centering
\begin{minipage}{.5\textwidth}
\centering
\begin{tikzpicture}[every node/.style={circle, draw, minimum size=0.05cm,fill=black}]
\node (v1) at (2,1) {};
\node (v2) at (0,0) {};
\node (v6) at (-2.5,-1.5) {};
\node (v3) at (2.5,-1.5) {};
\node (v5) at (-1.5,-4) {};
\node (v4) at (1.5,-4) {};
\draw[ultra thick]  (v1) edge (v2);
\draw  (v2) edge (v3);
\draw  (v2) edge (v4);
\draw[ultra thick]  (v3) edge (v4);
\draw  (v4) edge (v5);
\draw[ultra thick]  (v5) edge (v6);
\draw  (v6) edge (v2);
\end{tikzpicture}
\caption*{$\beta (G)=3$, $\gamma (G)=3$}
\end{minipage}%
\begin{minipage}{.5\textwidth}
\centering
\begin{tikzpicture}[every node/.style={circle, draw, minimum size=0.05cm,fill=black}]
\node (v1) at (0,-2) {};
\node (v2) at (0,0) {};
\node (v6) at (-2.5,-1.5) {};
\node (v3) at (2.5,-1.5) {};
\node (v5) at (-1.5,-4) {};
\node (v4) at (1.5,-4) {};

\draw[ultra thick]  (v2) edge (v1);
\draw  (v1) edge (v3);
\draw  (v4) edge (v1);
\draw  (v1) edge (v5);
\draw  (v6) edge (v1);
\end{tikzpicture}
\caption*{$\beta (G)=1$, $\gamma (G)=1$}
\end{minipage}
\end{figure}

\begin{definition}[$\beta ^\star (G)$]
Dla każdej krawędzi $C\subseteq G$ wprowadzamy zmienną $x_C$ taką, że $x_C=1$ gdy $e$ należy do skojarzenia, a $x_e=0$ gdy nie należy do skojarzenia:
$$f(x_1,\dots x_n)=\sum _{c\in E}x_c$$
($x_c$ $\rightarrow$ wartości na krawędziach) dla każdego wierzchołka $v$ 
$$\forall _v \sum _{v\in e}  x_e \leq 1$$
\begin{align*}
&x_e \in \xcancel{\{0,1\}}\\
&0\leq x_e\leq 1\\
&\beta ^\star (G)\ \rightarrow \text{ułamkowa liczba skojarzenia grafu}
\end{align*}
\end{definition}

\begin{fact}
$$\beta ^\star (G)\geq \beta (G)$$
\end{fact}

Dla każdego wierzchołka wprowadzamy zmienną $y_v$.\\Będziemy szukali minimum następującej funkcji
\begin{align*}
&g(y_1,\dots y_n)=\sum _{v\in V}y_v\\
&\forall _{\begin{matrix}
e\in E\\e\in\{v,w\}
\end{matrix}} y_v+y_w\geq 1\\
&y_v=[0;1]
\end{align*}
\begin{definition}[$\gamma (G)$]
Minimalna liczna pokryciowa $\gamma (G)$, to najmniejsza liczba wierzchołków w zbiorze $w\subseteq V$ takim, że każda krawędź ma przynajmniej jeden koniec w $w$.
\end{definition}

\begin{fact}
$$\gamma (G) \geq \beta (G)$$
\end{fact}
\begin{fact}
$$\gamma (G) \geq \gamma ^\star (G)$$
\end{fact}
\begin{fact}
$$\gamma (G) \geq \gamma ^\star (G)=\beta ^\star (G)\geq \beta (G)$$
\end{fact}

\begin{example*}
przykład:
\begin{figure}[H]
\centering
\begin{minipage}{.5\textwidth}
\centering
\begin{tikzpicture}%[main node/.style={circle, draw, minimum size=0.05cm,fill=black}]
\node (v2) at (0,0) {};
\node (v6) at (-2.5,-1.5) {};
\node (v3) at (2.5,-1.5) {};
\node (v5) at (-1.5,-4) {};
\node (v4) at (1.5,-4) {};
\draw  (v2) edge node[above] {$\sfrac{1}{2}$} (v3);
\draw  (v3) edge node[above] {$\sfrac{1}{2}$} (v4);
\draw  (v4) edge node[above] {$\sfrac{1}{2}$} (v5);
\draw  (v5) edge node[above] {$\sfrac{1}{2}$} (v6);
\draw  (v6) edge node[above] {$\sfrac{1}{2}$} (v2);
\end{tikzpicture}
\caption*{$\beta ^\star (G)\geq 2.5$}
\end{minipage}%
\begin{minipage}{.5\textwidth}
\centering
\begin{tikzpicture}[every node/.style={above,draw}]
\node (v2) at (0,0) {$\sfrac{1}{2}$};
\node (v6) at (-2.5,-1.5) {$\sfrac{1}{2}$};
\node (v3) at (2.5,-1.5) {$\sfrac{1}{2}$};
\node (v5) at (-1.5,-4) {$\sfrac{1}{2}$};
\node (v4) at (1.5,-4) {$\sfrac{1}{2}$};
\draw  (v2) edge (v3);
\draw  (v3) edge (v4);
\draw  (v4) edge (v5);
\draw  (v5) edge (v6);
\draw  (v6) edge (v2);
\end{tikzpicture}
\caption*{$\gamma ^\star (G)\leq 2.5$}
\end{minipage}
\end{figure}
$$\beta ^\star =\gamma ^\star$$
\end{example*}

Oczywiście:
$$\beta (G)\leq \floor{\frac{n}{2}}$$

\textbf{Zadanie:} Uzasadnij, że $$\beta ^\star (G) \leq \frac{n}{2}$$
\begin{proof}
$$\beta ^\star (G)=\gamma ^\star (G) \leq \frac{n}{2}$$
Bo każdemu wierzchołkowi możemy dać wagę $\frac{1}{2}$.
\end{proof}

\begin{figure}[H]
\centering
\begin{minipage}{.5\textwidth}
\centering
\begin{tikzpicture}%[every node/.style={circle, draw, minimum size=0.05cm,fill=black}]
\node (v1) at (2,1) {0};
\node (v2) at (0,0) {1};
\node (v6) at (-2.5,-1.5) {1};
\node (v3) at (2.5,-1.5) {0};
\node (v5) at (-1.5,-4) {0};
\node (v4) at (1.5,-4) {1};
\draw[ultra thick]  (v1) edge (v2);
\draw  (v2) edge (v3);
\draw  (v2) edge (v4);
\draw[ultra thick]  (v3) edge (v4);
\draw  (v4) edge (v5);
\draw[ultra thick]  (v5) edge (v6);
\draw  (v6) edge (v2);
\end{tikzpicture}
\end{minipage}%
\begin{minipage}{.5\textwidth}
\centering
\begin{tikzpicture}%[every node/.style={circle, draw, minimum size=0.05cm,fill=black}]
\node (v1) at (2,1) {};
\node (v2) at (0,0) {};
\node (v6) at (-2.5,-1.5) {};
\node (v3) at (2.5,-1.5) {};
\node (v5) at (-1.5,-4) {};
\node (v4) at (1.5,-4) {};
\draw  (v1) edge node[above] {1} (v2);
\draw  (v2) edge node[above] {0} (v3);
\draw  (v2) edge node[above] {1} (v4);
\draw  (v3) edge node[above] {1} (v4);
\draw  (v4) edge node[above] {0} (v5);
\draw  (v5) edge node[above] {1} (v6);
\draw  (v6) edge node[above] {0} (v2);
\end{tikzpicture}
\end{minipage}
\end{figure}

\begin{theorem}[D. Konig 1931]
Dla grafów dwudzielnych:
$$\begin{array}{ccc}
\beta (G) & = & \gamma (G) \\
\parallel &   & \parallel \\
\beta ^\star (G) & = & \gamma ^\star (G)
\end{array}$$
\end{theorem}

\begin{definition}[$\omega ^\star (G)$]
Ułamkową liczbą klikową $\omega ^\star (G)$ nazywamy maksymalną wartość funkcji:
$$f(x_1,\dots x_n)=\sum _{v\in V} x_v$$ 
przy czym dla każdego zbioru niezależnego $I$
\begin{align*}
&\sum _{v\in I} x_v\\
&x_v\geq 0\ \ x_v\in \{0,1\}
\end{align*}
\end{definition}

\begin{definition}[Ułamkowa liczba chromatyczna $\chi ^\star (G)$]
Każdemu zbiorowi niezależnemu $I$ przyporządkowujemy zmienną losową $y_I$ i minimalizujemy: $$\sum _I y_I$$ przy ograniczeniu, że 
\begin{align*}
&\forall _{v\in V} \sum _{\{I:v\in I\}}y_I\geq 1\\
&y_I\geq 0\\
&\text{dla }\chi (G) \text{ będzie } y_I=\{0,1\}
\end{align*}
\end{definition}

\begin{fact}
$$\chi (G)\geq \chi ^\star(G)=\omega ^\star(G)\geq \omega (G)$$
\end{fact}

\begin{figure}[H]
\centering
\begin{tikzpicture}%[main node/.style={circle, draw, minimum size=0.05cm,fill=black}]
\node (v2) at (0,0) {$v_1$};
\node (v6) at (-2.5,-1.5) {$v_5$};
\node (v3) at (2.5,-1.5) {$v_2$};
\node (v5) at (-1.5,-4) {$v_4$};
\node (v4) at (1.5,-4) {$v_3$};
\draw  (v2) edge (v3);
\draw  (v3) edge (v4);
\draw  (v4) edge (v5);
\draw  (v5) edge (v6);
\draw  (v6) edge (v2);
\end{tikzpicture}
\end{figure}
\begin{align*}
&v_1=v_2=v_3=v_4=v_5=\frac{1}{2}\\
&\text{zbiory niezależne: } 
\underset{\sfrac{1}{2}}{\{v_1,v_3\}}, \underset{\sfrac{1}{2}}{\{v_2,v_4\}},
\underset{\sfrac{1}{2}}{\{v_3,v_5\}}, \underset{\sfrac{1}{2}}{\{v_4,v_1\}},
\underset{\sfrac{1}{2}}{\{v_5,v_2\}}\\
&\chi (G)=3 & \omega (G) = 2\\
&\chi ^\star(G)\leq \sfrac{5}{2} & \omega ^\star(G)\geq \sfrac{5}{2}\\
&\chi ^\star(G) = \omega ^\star(G)= \sfrac{5}{2}\\
&\text{Ta własność } \chi (G)=\omega (G) \text{ jest bardzo rzadka wśród grafów}
\end{align*}






\end{document}
