\documentclass[a4paper,12pt]{article}

%% Language and font encodings
\usepackage[T1]{fontenc}
\usepackage[polish]{babel}
\usepackage[utf8]{inputenc}
\usepackage{lmodern}
\selectlanguage{polish}

%% Sets page size and margins
%\usepackage[a4paper,top=2cm,bottom=2cm,left=2cm,right=4cm,asymmetric]{geometry}
\usepackage{geometry}


%% Useful packages
\usepackage[fleqn]{amsmath}%[fleqn]
\usepackage{xfrac} %\sfrac{}{}

\usepackage{titlesec}%titles
\titlelabel{\thetitle.\quad}
\let\savenumberline\numberline
\def\numberline#1{\savenumberline{#1.}}
\usepackage{etoolbox}%dots in TOC
\makeatletter
\patchcmd{\l@section}
{\hfil}
{\leaders\hbox{\normalfont$\m@th\mkern \@dotsep mu\hbox{.}\mkern \@dotsep mu$}\hfill}
{}{}
\makeatother

\usepackage{caption,graphicx}
\usepackage{float}
\usepackage{sidecap}
\usepackage[colorinlistoftodos]{todonotes}
\usepackage[colorlinks=true, allcolors=blue]{hyperref}

\usepackage{tikz}
\usepackage{tikz-qtree}
\usetikzlibrary{trees}
\usetikzlibrary{arrows,positioning,shapes,fit,calc,decorations.pathreplacing}
\usetikzlibrary{graphs}
\usetikzlibrary{graphs.standard}
\usepackage{forest}
\usepackage{tikzscale}
\usepackage{pgfgantt} %Diagramy gantta http://bay.uchicago.edu/CTAN/graphics/pgf/contrib/pgfgantt/pgfgantt.pdf
\usepackage{pgf}
\usepackage{caption}

\usepackage{fancybox}
\usepackage{listings}

\usepackage[colorinlistoftodos]{todonotes} %http://mirror.unl.edu/ctan/macros/latex/contrib/todonotes/todonotes.pdf

\usepackage{array,longtable}

\newcommand\floor[1]{\lfloor#1\rfloor} %PODŁOGA -> \floor
\newcommand\ceil[1]{\lceil#1\rceil} %SUFIT -> \ceil

\usepackage{fancyhdr}
\pagestyle{fancy}
\rhead{\thepage}
\lhead{\leftmark}
\rfoot{\thepage}
\lfoot{\rightmark}

%http://tex.stackexchange.com/questions/64170/which-package-to-use-for-writing-algorithms
\usepackage{algorithm}% http://ctan.org/pkg/algorithms
\usepackage{algpseudocode}% http://ctan.org/pkg/algorithmicx
\newcommand{\var}[1]{{\ttfamily#1}}% variable

\algnewcommand\algorithmicforeach{\textbf{for each}} %Algorithm foreach
\algdef{S}[FOR]{ForEach}[1]{\algorithmicforeach\ #1\ \algorithmicdo}

\usepackage{amsthm}
\usepackage[inline]{enumitem} %enumerations
\usepackage{multicol}

\theoremstyle{definition}%~ %%% <-  Note that space!
\newtheorem{lemma}{Lemat} %\begin{lemma} ... \end{lemma} LEMAT(?)
\newtheorem{remark}{Wniosek}%\begin{remark} ... \end{remark} WNIOSEK 
%\newtheorem*{remark*}{Wniosek}%\begin{remark} ... \end{remark} WNIOSEK bez liczby
\newtheorem{theorem}{Twierdzenie}%\begin{theorem} ... \end{theorem}
\newtheorem{fact}{Fakt} %\begin{fact} ... \end{fact}
\newtheorem*{fact*}{Fakt} %\begin{fact*} ... \end{fact*} Fakt
\newtheorem*{observation*}{Obserwacja}

\newtheorem{example}{Przykład}
\newtheorem*{example*}{Przykład} %\begin{example} ... \end{example}
\theoremstyle{definition}
\newtheorem{definition}{Definicja}%\begin{definition}{} ... \end{definition}
%\newtheorem*{definition*}{Definicja}
%\newtheorem*{hipoterm*}{Hipoteza}%\begin{hipoterm*}[] ... \end{hipoterm*}
\newtheorem{hipoterm}{Hipoteza}%\begin{hipoterm*}[] ... \end{hipoterm*}
\theoremstyle{problem}
\newtheorem{problem}{Problem}%\begin{problem}{} ... \end{problem}
\newtheorem*{problem*}{Problem}

\let\originalforall=\forall%FORALL
\renewcommand{\forall}{\mathop{\vcenter{\hbox{\Large$\originalforall$}}}}
\let\originalexists=\exists%EXISTS
\renewcommand{\exists}{\mathop{\vcenter{\hbox{\Large$\originalexists$}}}}

\usepackage{cancel} %skreślenie równania \xcancel{...} \cancel{} lub \bcancel{}

\usepackage{amsfonts}

\usepackage{comment}

\usepackage{xcolor,colortbl}
\usepackage{multirow}

%\usepackage{wrapfig} %wrap text around figure

\usepackage{pdfpages}%\includepdf{file}

\usepackage{etoolbox}
\let\bbordermatrix\bordermatrix
\patchcmd{\bbordermatrix}{8.75}{4.75}{}{}
\patchcmd{\bbordermatrix}{\left(}{\left[}{}{}
\patchcmd{\bbordermatrix}{\right)}{\right]}{}{}
%\bbordermatrix{}

\allowdisplaybreaks

\title{Struktury Dyskretne - Notatki}
\author{Piotr Parysek\\
\href{mailto:piotr.parysek@outlook.com}{piotr.parysek@outlook.com} }
\date{\today}

\begin{document}
\maketitle

\tableofcontents
\section[Wykład 9: 4-V-2017 - Temat: Łańcuchy Markowa - Losowe generowanie obiektów]{Temat: Łańcuchy Markowa - Losowe generowanie obiektów}
\subsection{Łańcuchy odwracalne}
\begin{definition}[Odwracalność Łańcuchów Markowa]\label{def:OdwracalnoscLM}
Łańcuch Markowa jest odwracalny, gdy istnieje wektor $\bar{a}=(\bar{a_1}, ..., \bar{a_S})$ taki, że dla każdej pary $i,j$: $$a_i\pi _{ij}=a_j\pi _{ji}$$  wtedy wektor $\bar{\pi}=(\pi _1,...,\pi _S)$ taki, że $$\pi _i=\frac{a_i}{\sum _{s\in S}a_i}$$ jest wektorem stacjonarnym Łańcuchu Markowa.
\end{definition}
\begin{remark}
Jeśli dla dowolnej pary stanów $i,j\in S$ mamy: $$\pi_{i,j}=\pi_{j,i}$$ to wektor $$\bar{\pi}=\left(\frac{1}{|S|},\frac{1}{|S|},...,\frac{1}{|S|}\right)$$ jest rozkładem stacjonarnym Łańcucha Markowa
\end{remark}
\begin{definition}[Odwracalność Łańcuchów Markowa]\label{def:OdwracalnoscLM2}
Rozkład $\bar{\pi}$ nazywamy \textbf{odwracalnym}, gdy: $$\forall _{i,j\in S} \pi _i p_{i,j}=\pi _j p_{j,i}$$ gdzie $\pi _i$ jest i-tą współrzędną wektora przwdopodobieństwa rozkładu $\bar{\pi}$, a $p_{i,j}$ prawdopodobieństwem przejścia ze stanu $s_i$ do stanu $s_j$.

Definicja pobrana z dokumentu: \url{http://www.staff.amu.edu.pl/~rucinski/rap412/rap412.pdf}.
\end{definition}

\subsection{ŁM - zbiory niezależne}
\textbf{Łańcuch Markowa działający na zbiorach niezależnych grafu $G$}
\begin{enumerate}
\item Startujemy od zbioru pustego $I=\emptyset $
\item Rzucamy symetryczną monetą otrzymując \textbf{O}rła albo \textbf{R}eszkę,
\item Wybieramy losowo wierzchołek $v$ w grafie $G$,
\item Jeśli 
\begin{itemize}
\item[] wypadł \textbf{O}rzeł i $I=I\cup \{v\}$ jest niezależny to\\
$I:=I\cup \{v\}$
\item[] wypadła \textbf{R}eszka i $v\in I$ to\\
$I:=I- \{v\}$
\end{itemize}
\item wracamy do punktu 2.
\end{enumerate}
\begin{itemize}
\item Czy ten Łańcuch Markowa jest okresowy?\\\textbf{NIE}
\item Czy ten Łańcuch Markowa jest nieprzywiedlny?\\\textbf{TAK}
\end{itemize}
Łańcuch Markowa jest ergodyczny, czyli dąży do swojego jedynego rozkładu stacjonarnego

\begin{figure}[H]
\centering
\begin{tikzpicture}[font=\boldmath,ultra thick,]
\draw[solid, line width=1mm]  (0,0) ellipse (2 and 2) node [above=2cm] {$I$};
\draw[solid, line width=1mm]  (6,0) ellipse (2 and 2) node [above=2cm] {$J=I\cup \{v\}$};
\node (v1) at (1.5,-2) {};
\node (v2) at (4.5,-2) {};
\node (v3) at (1.5,-3) {};
\node (v4) at (4.5,-3) {};
\draw[->,solid,line width=1mm,fill=black]  (v1) edge node [above] {$\pi _{i,j}=\frac{1}{2}*\frac{1}{n}$} (v2);
\draw[->,solid,line width=1mm,fill=black]  (v4) edge node [above] {$\pi _{j,i}=\frac{1}{2}*\frac{1}{n}$}(v3);
\end{tikzpicture}
\caption*{$\frac{1}{n}$ $\rightarrow$ wybranie 1 z $n$ wierzchołków\\$\frac{1}{2}$ $\rightarrow$ \textbf{O}rzeł albo \textbf{R}eszka}
\end{figure}
\begin{remark}
Jedynym rozkładem stacjonarnym tego Łańcucha Markowa jest rozkład jednostajny.
\end{remark}
\begin{problem*}
Skonstruować Łańcuch Markowa taki, żeby prawdopodobieństwo każdego ze zbioru niezależnych $I$ było proporcjonalne do $$\alpha ^I$$ gdzie $\alpha $ jest pewną stałą.
\end{problem*}

\subsubsection{Modyfikacja algorytmu}
\begin{enumerate}
\item Startujemy od zbioru pustego $I=\emptyset $
\item Rzucamy monetą taką, że $\mathsf{Pr}(O)=p$\footnote{prawdopodobieństwo ,,wylosowania'' \textbf{O}rła} a $\mathsf{Pr}(R)=1-\mathsf{Pr}(O)=1-p$\footnote{prawdopodobieństwo ,,wylosowania'' \textbf{R}eszki}
\item Wybieramy losowo wierzchołek $v$ w grafie $G$,
\item Jeśli 
\begin{itemize}
\item[] wypadł \textbf{O}rzeł i $I=I\cup \{v\}$ jest niezależny to\\
$I:=I\cup \{v\}$
\item[] wypadła \textbf{R}eszka i $v\in I$ to\\
$I:=I- \{v\}$
\end{itemize}
\item wracamy do punktu 2.
\end{enumerate}

\begin{align*}
\alpha ^{|I|}\pi_{i,j}&=\alpha ^{|J|}\pi_{j,i}\\
\alpha ^{|I|}p\frac{1}{n}=\alpha ^{|J|}(1-p)\frac{1}{n}&=\alpha *\alpha ^{|I|}(1-p)\frac{1}{n}\\
p&=\alpha (1-p)\\
p&=\alpha -\alpha p\\
p+\alpha p&=\alpha\\
p&=\frac{\alpha}{1+\alpha}
\end{align*}

\begin{hipoterm}[Metatwierdzenie (Obserwacja)]
Jeśli potrafimy wygenerować losowy obiekt z danej klasy (być w możę w przybliżeniu) to potrafimy te obiekty (z grubsza) policzyć
\end{hipoterm}

\begin{problem*}
Policz zbiory niezależne w danym grafie $G$
\begin{enumerate}
\item Mamy graf $G$
\item Wybieramy dowolny wierzchołek $v\in V(G)$
\item Szacujemy prawdopodobieństwo $p_v$, że $v$ należy do niezależnego, to znaczy że $v$ należy do $\mathsf{pr}|S_G|$ zbiorów niezależnych \
\item Gdzie $S_G$ jest rodziną wszystkich zbiorów niezależnych w $G$
\begin{enumerate}[label=\Alph*)]
\item Jeśli $p_v\geq \frac{1}{2}$ wtedy $$G'=G- \{v\}-N(v)$$ wtedy $\mathsf{pr}|S_G|-|S_{G'}|$ wybieramy $v'\in V(G')$ i powtarzamy całą procedurę
\item Jeśli $p_v < \frac{1}{2}$ wtedy $$G'=G- \{v\}$$ wybieramy $v'\in V(G')$ i powtarzamy całą procedurę wtedy $$(1-p_v)|S_G|=|S_{G'}|$$
\end{enumerate}
\end{enumerate}
$$\underset{A}{p_{v_1}},\underset{B}{p_{v_2}},\underset{B}{p_{v_3}},...,\underset{B}{p_{v_n}}$$
\begin{figure}[H]
\centering
\begin{tikzpicture}
\draw[solid, line width=1mm]  (0,0) ellipse (1 and 1) node [above=1cm] {$n$}
node [below=1cm] {$|S_{G'}|$};
\end{tikzpicture}
\end{figure}
\begin{align*}
(1-p_{v_n})|S_{G_{n-1}}|&=|S_{G_n}|\\
|S_{G_{n-1}}|&=\frac{|S_{G_n}|}{1-p_{v_n}}\\
|S_{G_n}|&=\frac{|S_{G_n}|}{\prod_{i=i}^{n-1}\rho _i}\\
\rho _i &= \left\{\begin{matrix}
p_i & \text{ gry zawarte w } A\\
1-p_i &  \text{ gry zawarte w } B
\end{matrix}\right.\\
v_1\ p_{v_1}\geq \frac{1}{2}\ &\ G_2=\{v_1\}\cup N(v_1)\\
v_2\ p_{v_2}\frac{1}{2}\ &\ G_3=g_2- \{v_2\}\\
(1-p_{v_2})|S_{G_2}|&=|S_{G_3}|\\
|S_{G_2}|=\frac{|S_{G_3}|}{1-p_{v_2}}\\
p_{v_1}|S_{G_1}|=|S_{G_2}|&=\frac{|S_{G_3}|}{1-p_{v_2}}\\
|S_{G_1}|&=\frac{|S_{G_3}|}{p_{v_1}(1-p_{v_2})}
\end{align*}
\end{problem*}


\end{document}
