\documentclass[a4paper,12pt]{article}

%% Language and font encodings
\usepackage[T1]{fontenc}
\usepackage[polish]{babel}
\usepackage[utf8]{inputenc}
\usepackage{lmodern}
\selectlanguage{polish}

%% Sets page size and margins
%\usepackage[a4paper,top=2cm,bottom=2cm,left=2cm,right=4cm,asymmetric]{geometry}
\usepackage{geometry}


%% Useful packages
\usepackage[fleqn]{amsmath}%[fleqn]
\usepackage{xfrac} %\sfrac{}{}

\usepackage{titlesec}%titles
\titlelabel{\thetitle.\quad}
\let\savenumberline\numberline
\def\numberline#1{\savenumberline{#1.}}
\usepackage{etoolbox}%dots in TOC
\makeatletter
\patchcmd{\l@section}
{\hfil}
{\leaders\hbox{\normalfont$\m@th\mkern \@dotsep mu\hbox{.}\mkern \@dotsep mu$}\hfill}
{}{}
\makeatother

\usepackage{caption,graphicx}
\usepackage{float}
\usepackage{sidecap}
\usepackage[colorinlistoftodos]{todonotes}
\usepackage[colorlinks=true, allcolors=blue]{hyperref}

\usepackage{tikz}
\usepackage{tikz-qtree}
\usetikzlibrary{trees}
\usetikzlibrary{arrows,positioning,shapes,fit,calc,decorations.pathreplacing}
\usetikzlibrary{graphs}
\usetikzlibrary{graphs.standard}
\usepackage{forest}
\usepackage{tikzscale}
\usepackage{pgfgantt} %Diagramy gantta http://bay.uchicago.edu/CTAN/graphics/pgf/contrib/pgfgantt/pgfgantt.pdf
\usepackage{pgf}
\usepackage{caption}

\usepackage{fancybox}
\usepackage{listings}

\usepackage[colorinlistoftodos]{todonotes} %http://mirror.unl.edu/ctan/macros/latex/contrib/todonotes/todonotes.pdf

\usepackage{array,longtable}

\newcommand\floor[1]{\lfloor#1\rfloor} %PODŁOGA -> \floor
\newcommand\ceil[1]{\lceil#1\rceil} %SUFIT -> \ceil

\usepackage{fancyhdr}
\pagestyle{fancy}
\rhead{\thepage}
\lhead{\leftmark}
\rfoot{\thepage}
\lfoot{\rightmark}

%http://tex.stackexchange.com/questions/64170/which-package-to-use-for-writing-algorithms
\usepackage{algorithm}% http://ctan.org/pkg/algorithms
\usepackage{algpseudocode}% http://ctan.org/pkg/algorithmicx
\newcommand{\var}[1]{{\ttfamily#1}}% variable

\algnewcommand\algorithmicforeach{\textbf{for each}} %Algorithm foreach
\algdef{S}[FOR]{ForEach}[1]{\algorithmicforeach\ #1\ \algorithmicdo}

\usepackage{amsthm}
\usepackage[inline]{enumitem} %enumerations
\usepackage{multicol}

\theoremstyle{definition}%~ %%% <-  Note that space!
\newtheorem{lemma}{Lemat} %\begin{lemma} ... \end{lemma} LEMAT(?)
\newtheorem{remark}{Wniosek}%\begin{remark} ... \end{remark} WNIOSEK 
%\newtheorem*{remark*}{Wniosek}%\begin{remark} ... \end{remark} WNIOSEK bez liczby
\newtheorem{theorem}{Twierdzenie}%\begin{theorem} ... \end{theorem}
\newtheorem{fact}{Fakt} %\begin{fact} ... \end{fact}
\newtheorem*{fact*}{Fakt} %\begin{fact*} ... \end{fact*} Fakt
\newtheorem*{observation*}{Obserwacja}

\newtheorem{example}{Przykład}
\newtheorem*{example*}{Przykład} %\begin{example} ... \end{example}
\theoremstyle{definition}
\newtheorem{definition}{Definicja}%\begin{definition}{} ... \end{definition}
%\newtheorem*{definition*}{Definicja}
%\newtheorem*{hipoterm*}{Hipoteza}%\begin{hipoterm*}[] ... \end{hipoterm*}
\newtheorem{hipoterm}{Hipoteza}%\begin{hipoterm*}[] ... \end{hipoterm*}
\theoremstyle{problem}
\newtheorem{problem}{Problem}%\begin{problem}{} ... \end{problem}
\newtheorem*{problem*}{Problem}

\let\originalforall=\forall%FORALL
\renewcommand{\forall}{\mathop{\vcenter{\hbox{\Large$\originalforall$}}}}
\let\originalexists=\exists%EXISTS
\renewcommand{\exists}{\mathop{\vcenter{\hbox{\Large$\originalexists$}}}}

\usepackage{cancel} %skreślenie równania \xcancel{...} \cancel{} lub \bcancel{}

\usepackage{amsfonts}

\usepackage{comment}

\usepackage{xcolor,colortbl}
\usepackage{multirow}

%\usepackage{wrapfig} %wrap text around figure

\usepackage{pdfpages}%\includepdf{file}

\usepackage{etoolbox}
\let\bbordermatrix\bordermatrix
\patchcmd{\bbordermatrix}{8.75}{4.75}{}{}
\patchcmd{\bbordermatrix}{\left(}{\left[}{}{}
\patchcmd{\bbordermatrix}{\right)}{\right]}{}{}
%\bbordermatrix{}

\allowdisplaybreaks

\title{Struktury Dyskretne - Notatki}
\author{Piotr Parysek\\
\href{mailto:piotr.parysek@outlook.com}{piotr.parysek@outlook.com} }
\date{\today}

\begin{document}
\maketitle

\tableofcontents
\section[Wykład 10: 18-V-2017 - Temat: Teoria kodów]{Temat: Teoria kodów}
\subsection{Podstawowe definicje}
\begin{description}
\item[$Q$] alfabet $$Q=\{0,1,2,\dots ,q-1\}$$
$|Q|=q$ $\rightarrow$ liczba elementów alfabetu.\\
Na przykład:\\
$Q=\{0,1\}\rightarrow$ kody binarne, gdzie $|Q|=q=2$
\item[$n$] długość słowa
\end{description}
inne przykłady kodów:
\begin{itemize}
\item Kod ASCII $0-127\rightarrow 2^7$ gdzie ostatni bit to bit parzystości (mowa o historii)
\item PESEL, przykłady:
\begin{itemize}
\item[] $11304900014\rightarrow$ zły kod PESEL - dzień miesiąca 49? 
\item[] $11300900014\rightarrow$ niepoprawny kod PESEL
\item[] $11304900014\rightarrow$ kom mogący być kodem PESEL
\end{itemize}
\end{itemize}

\subsection{Teoria kodów:}
\begin{observation*}
Istnieje $q^n$ różnych słów długości $n$ nad alfabetem $Q$, $|Q|=q$.
\end{observation*}
\begin{definition}[Kod]
Kodem $C$ o długości $n$ nad alfabetem $Q$ nazywamy dowolny podzbiór o długości $n$ którego literami są elementy alfabetu $Q$ co zapisujemy: $$C\subseteq Q^n$$
Przykładowo słowo $\bar{x}=010111$ będziemy traktować jako wektor (kolumnowy) o argumentach $010111$, czyli: $$\bar{x}=\begin{bmatrix}
0\\1\\0\\1\\1\\1
\end{bmatrix}$$
\end{definition}
\begin{definition}[Odległość Hamminga]
W teorii kodów wyznaczamy \textbf{odległość Hamminga} to znaczy: $$d(\bar{x},\bar{y})=|\{j:x_i\neq y_i\}|$$
\begin{multicols}{2}
$$\bar{x}=0,1,2,5,7,1$$
$$\bar{y}=0,1,1,5,6,1$$
$$\bar{z}=7,1,2,5,7,1$$
\vfill\null
\columnbreak
$$d(\bar{x},\bar{y})=2$$
$$d(\bar{x},\bar{z})=1$$
\end{multicols}
\end{definition}

\begin{definition}[Rozstęp kodu]
\textbf{Rozstępem kodu} $C$ nazywamy najmniejszą odległość pomiędzy wyrazami tego kodu: $$r(C)=\min _{\begin{matrix}
\bar{x},\bar{y}\in C\\
\bar{x}\neq \bar{y}
\end{matrix}}d(\bar{x},\bar{y})$$
\end{definition}
\begin{observation*}
Jeśli chcemy by kod $C$ był odporny na zakłócenia komunikacji powudujące co najwyżej $r$ błędów to: $$r(C)=2R+1$$
\end{observation*}
\begin{problem*}
Oszacować liczbę wyrazów w kodzie $C$ o długości $n$ i rozstępie $r(C)=2R+1$
\end{problem*}
\subsection{Oszacowanie objętościowe}
\begin{itemize}
\item Weźmy $\bar{x}$
\item Ile jest wyrazów $\bar{y}$ takich, że $d(\bar{x},\bar{y})\leq r$?\\
Ile jest wyrazów w kuli o środku w $\bar{x}$ i promieniu $r$?\footnote{\#Shame: $\binom{n}{k}=\frac{n!}{k!(n-k)!}$}
\begin{align*}
&\text{Dane: }n\ \bar{x}=00\dots 0,\ Q=\{0,1,\dots ,q-1\}\\
&|\{\bar{y}:d(\bar{x},\bar{y})=0\}|=1\\
&|\{\bar{y}:d(\bar{x},\bar{y})=1\}|=n(q-1)\\
&|\{\bar{y}:d(\bar{x},\bar{y})=2\}|=\binom{n}{2}(q-1)(q-1)\\
&|\{\bar{y}:d(\bar{x},\bar{y})=i\}|=\binom{n}{i}(q-1)^i\\
&\sum _{i=0}^R\binom{n}{i}(q-1)^i\\
&|C|\sum _{i=0}^R\binom{n}{i}(q-1)^i\leq q^n\\
&|C|\leq  q^n\left(\sum _{i=0}^R\binom{n}{i}(q-1)^i\right)^{-1}
\end{align*}
\end{itemize}
\begin{definition}[Kod doskonały]
Kod $C$ o długości $n$ i odstępie $r(C)=2R+1$ nazywamy \textbf{doskonałym} gdy: $$|C|\sum _{i=0}^R\binom{n}{i}(q-1)^i= q^n$$
\end{definition}
\begin{example*}
4 kod ternarny $C_4(9)$ składający się z 9 słów: $$0000,0111,0222,1021,2012,1102,2201,2120,1210$$ jest kodem doskonałym o rozstępie 3:
\begin{align*}
&n=4,\ q=3,\ r=1,\ |C|=9\\
&9*\sum_{i=0}^R(q-1)^i=9*\left(1*2^0+4*2^1\right)=9*9=3^4
\end{align*}
\end{example*}

\subsection{Kody liniowe}
\begin{definition}[Kody Liniowe]
Niech $F_q$ będzie ciałem o $q$ elementach.\\ Takie ciało istnieje wtedy i tylko wtedy, gdy $q$ jest potęgą liczby pierwszej $$q=p^n$$ $p\rightarrow$ liczba pierwsza.\\ Na naszych zajęciach zawsze będziemy zakładać, że $q$ jest liczbą pierwszą.
\begin{align*}
&q\rightarrow\text{Liczba pierwsza}\\
&F_q=\{0,1,2,\dots ,q-1\}\\
&\text{Operacje: }\\
&&+ (\mod q)\\
&&* (\mod q)
\end{align*}
\end{definition}
\begin{example*}
Dla $F_7$ załóżmy, że mamy równanie:
\begin{align*}
&6x=2\\
&6x\equiv  2 (\mod 7)\\
&x \equiv 5
\end{align*}
\end{example*}

Jeśli mamy kilka wektorów $\bar{x}_1,\bar{x}_2,\dots ,\bar{x}_n\in Q^n=F_q^n$
$$\alpha _1 \bbordermatrix{&\bar{x}_1\cr
&\vdots\cr &\vdots\cr &\vdots}+\alpha _2 \bbordermatrix{&\bar{x}_2\cr
&\vdots\cr &\vdots\cr &\vdots}+\dots +\alpha _n \bbordermatrix{&\bar{x}_n\cr
&\vdots\cr &\vdots\cr &\vdots}=\bbordermatrix{&\bar{y}\cr
&\vdots\cr &\vdots\cr &\vdots}$$

\begin{definition}[ n,k kod]
Kodem $[n,k]$ będziemy nazywali dowolną podprzestrzeń liniową $C$ wymiaru $k$ w $F_q^n=Q^n$.\\$[n,k]$ kod $C$ nazywamy $[n,k,d]$ kodem jeśli $r(C)=d$
\end{definition}

\begin{definition}[Wektor bazowy]
Niech $\bar{x}_1,\bar{x}_2,\dots ,\bar{x}_k$ będą wektorami bazowymi $C$, to znaczy każdy element $\bar{y}\in C$ da się przedstawić w sposób jednoznaczny jako kombinację $\bar{x}_1,\bar{x}_2,\dots ,\bar{x}_k$ to znaczy istnieją $\alpha _1,\alpha _2,\dots ,\alpha _k \in Q$ takie, że:
$$\bar{y}=\alpha _1\bar{x}_1+\alpha _2\bar{x}_2+\dots +\alpha _k\bar{x}_k$$
\textbf{Pytanie:} ile elementów ma $[n,k]$ kod?\\
\textbf{Odpowiedź:} $$\underbrace{q*q*\dots *q}_k=q^k$$
\end{definition}
\begin{example*}[Przykład z kodu termanego]
9 słów, $Q=3$, $9=3^2$ $[4,2]$, $|C|=9$
$$\bar{y}=\alpha _1\begin{bmatrix}
0\\1\\1\\1
\end{bmatrix}+\alpha _2\begin{bmatrix}
1\\0\\2\\1
\end{bmatrix}\Leftrightarrow \begin{bmatrix}
2\\1\\2\\0
\end{bmatrix}=1\begin{bmatrix}
0\\1\\1\\1
\end{bmatrix}+2\begin{bmatrix}
1\\0\\2\\1
\end{bmatrix}$$
\end{example*}

\begin{definition}[Macierz generująca]\textbf{Macierzą generującą} $G$ $[n,k]$ kodu $C$ generowanego przez bazę $\bar{x}_1,\bar{x}_2,\dots ,\bar{x}_k$ nazywamy macierz, której wierszami są wektory: $\bar{x}_1^T,\bar{x}_2^T,\dots ,\bar{x}_k^T$

Dla przykładu powyżej:
$$G=\begin{bmatrix}
0&1&1&1\\
1&0&2&1
\end{bmatrix},\ G=\begin{bmatrix}
0&2&2&2\\
1&2&1&0
\end{bmatrix}$$
Zauważmy, że $$C=\{\alpha ^Y*G:\alpha ^T\in G^k\}$$
Dalej ten sam przykład:
$$[ 1, 2 ] \begin{bmatrix}
0&1&1&1\\
1&0&2&1
\end{bmatrix}=\begin{bmatrix}
2&1&2&0
\end{bmatrix}=1*\begin{bmatrix}
0&1&1&1
\end{bmatrix}+2\begin{bmatrix}
0&1&1&1
\end{bmatrix}$$
\end{definition}



\begin{definition}[Macierz parzystości]
\textbf{Macierzą parzystości} $H$ $[n,k]$ kodu $C$ jest macierz o rozmiarze $(n-k)\times n$ o rzędzie $n-k$ taka, że $$GH^T=O$$ gdzie $O$ oznacza macierz zerową. 
\end{definition}
\begin{fact*}
Słowo $\bar{y}\in C$ wtedy i tylko wtedy, gdy $K\bar{y}=0$
\end{fact*}

\begin{definition}[Ortogonalność]
W przestrzeni liniowej z iloczynem skalarnym $<\ >$ wektory $u,v\in V$ są \textbf{ortogonalne} (lub prostopadłe) gdy $<u,v>=0$. Piszemy wtedy $u\bot v$.
\end{definition}

\begin{definition}[Układ ortogonalny, układ ortonormalny]
Mówimy, że układ wektorów $(v_i)_{i\in I}$ z przestrzeni $V$ jest \textbf{układem ortogonalnym} gdy $<v_j, v_k> = 0$ dla $j\neq k$ . \textbf{Układ} jest \textbf{ortonormalny} gdy jest ortogonalny oraz $<v_j
, v_j> = 1$ dla $j\in I$.
\end{definition}


\end{document}
