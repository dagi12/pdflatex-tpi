\documentclass[a4paper,12pt]{article}

%% Language and font encodings
\usepackage[T1]{fontenc}
\usepackage[polish]{babel}
\usepackage[utf8]{inputenc}
\usepackage{lmodern}
\selectlanguage{polish}

%% Sets page size and margins
%\usepackage[a4paper,top=2cm,bottom=2cm,left=2cm,right=4cm,asymmetric]{geometry}
\usepackage{geometry}


%% Useful packages
\usepackage[fleqn]{amsmath}%[fleqn]
\usepackage{xfrac} %\sfrac{}{}

\usepackage{titlesec}%titles
\titlelabel{\thetitle.\quad}
\let\savenumberline\numberline
\def\numberline#1{\savenumberline{#1.}}
\usepackage{etoolbox}%dots in TOC
\makeatletter
\patchcmd{\l@section}
{\hfil}
{\leaders\hbox{\normalfont$\m@th\mkern \@dotsep mu\hbox{.}\mkern \@dotsep mu$}\hfill}
{}{}
\makeatother

\usepackage{caption,graphicx}
\usepackage{float}
\usepackage{sidecap}
\usepackage[colorinlistoftodos]{todonotes}
\usepackage[colorlinks=true, allcolors=blue]{hyperref}

\usepackage{tikz}
\usepackage{tikz-qtree}
\usetikzlibrary{trees}
\usetikzlibrary{arrows,positioning,shapes,fit,calc,decorations.pathreplacing}
\usetikzlibrary{graphs}
\usetikzlibrary{graphs.standard}
\usepackage{forest}
\usepackage{tikzscale}
\usepackage{pgfgantt} %Diagramy gantta http://bay.uchicago.edu/CTAN/graphics/pgf/contrib/pgfgantt/pgfgantt.pdf
\usepackage{pgf}
\usepackage{caption}

\usepackage{fancybox}
\usepackage{listings}

\usepackage[colorinlistoftodos]{todonotes} %http://mirror.unl.edu/ctan/macros/latex/contrib/todonotes/todonotes.pdf

\usepackage{array,longtable}

\newcommand\floor[1]{\lfloor#1\rfloor} %PODŁOGA -> \floor
\newcommand\ceil[1]{\lceil#1\rceil} %SUFIT -> \ceil

\usepackage{fancyhdr}
\pagestyle{fancy}
\rhead{\thepage}
\lhead{\leftmark}
\rfoot{\thepage}
\lfoot{\rightmark}

%http://tex.stackexchange.com/questions/64170/which-package-to-use-for-writing-algorithms
\usepackage{algorithm}% http://ctan.org/pkg/algorithms
\usepackage{algpseudocode}% http://ctan.org/pkg/algorithmicx
\newcommand{\var}[1]{{\ttfamily#1}}% variable

\algnewcommand\algorithmicforeach{\textbf{for each}} %Algorithm foreach
\algdef{S}[FOR]{ForEach}[1]{\algorithmicforeach\ #1\ \algorithmicdo}

\usepackage{amsthm}
\usepackage[inline]{enumitem} %enumerations
\usepackage{multicol}

\theoremstyle{definition}%~ %%% <-  Note that space!
\newtheorem{lemma}{Lemat} %\begin{lemma} ... \end{lemma} LEMAT(?)
\newtheorem{remark}{Wniosek}%\begin{remark} ... \end{remark} WNIOSEK 
%\newtheorem*{remark*}{Wniosek}%\begin{remark} ... \end{remark} WNIOSEK bez liczby
\newtheorem{theorem}{Twierdzenie}%\begin{theorem} ... \end{theorem}
\newtheorem{fact}{Fakt} %\begin{fact} ... \end{fact}
\newtheorem*{fact*}{Fakt} %\begin{fact*} ... \end{fact*} Fakt
\newtheorem*{observation*}{Obserwacja}

\newtheorem{example}{Przykład}
\newtheorem*{example*}{Przykład} %\begin{example} ... \end{example}
\theoremstyle{definition}
\newtheorem{definition}{Definicja}%\begin{definition}{} ... \end{definition}
%\newtheorem*{definition*}{Definicja}
%\newtheorem*{hipoterm*}{Hipoteza}%\begin{hipoterm*}[] ... \end{hipoterm*}
\newtheorem{hipoterm}{Hipoteza}%\begin{hipoterm*}[] ... \end{hipoterm*}
\theoremstyle{problem}
\newtheorem{problem}{Problem}%\begin{problem}{} ... \end{problem}
\newtheorem*{problem*}{Problem}

\let\originalforall=\forall%FORALL
\renewcommand{\forall}{\mathop{\vcenter{\hbox{\Large$\originalforall$}}}}
\let\originalexists=\exists%EXISTS
\renewcommand{\exists}{\mathop{\vcenter{\hbox{\Large$\originalexists$}}}}

\usepackage{cancel} %skreślenie równania \xcancel{...} \cancel{} lub \bcancel{}

\usepackage{amsfonts}

\usepackage{comment}

\usepackage{xcolor,colortbl}
\usepackage{multirow}

%\usepackage{wrapfig} %wrap text around figure

\usepackage{pdfpages}%\includepdf{file}

\usepackage{etoolbox}
\let\bbordermatrix\bordermatrix
\patchcmd{\bbordermatrix}{8.75}{4.75}{}{}
\patchcmd{\bbordermatrix}{\left(}{\left[}{}{}
\patchcmd{\bbordermatrix}{\right)}{\right]}{}{}
%\bbordermatrix{}

\allowdisplaybreaks

\title{Struktury Dyskretne - Notatki}
\author{Piotr Parysek\\
\href{mailto:piotr.parysek@outlook.com}{piotr.parysek@outlook.com} }
\date{\today}

\begin{document}
\maketitle

\tableofcontents
\section[Wykład 7: 20-IV-2017 - Temat: Łańcuchy Markowa]{Temat: Łańcuchy Markowa}
Mottem a zarazem prezentacją wykładu jest ,,gra w życie'' \url{https://bitstorm.org/gameoflife/}

\subsection{,,Rozgrzewka''}
\begin{example*}\label{exa:student}
Student błądzi między Biblioteką, Wykładem a Ćwiczeniami i co 5 minut decyduje co robić dalej?\\
Graf możliwych decyzji studenta poniżej:
\begin{figure}[H]
\centering
\begin{tikzpicture}[shorten >=1pt, auto, node distance=3cm, ultra thick,main node/.style={circle,draw,minimum size=.4cm,inner sep=0pt}]
\begin{scope}[every node/.style={font=\sffamily\large\bfseries}]
\node [main node](v1) at (2,3) {B};
\node [main node](v2) at (0,0) {C};
\node [main node](v3) at (4,0) {W};
\end{scope}
\path
(v1) edge [->] node {$\sfrac{1}{4}$} (v2)
edge [->,bend left] node {$\sfrac{1}{4}$} (v3)
edge [loop above]   node {$\sfrac{1}{2}$} (v1)
(v2) edge [->,bend left] node {$\sfrac{2}{3}$} (v1)
edge [->] node {$\sfrac{1}{3}$} (v3)
(v3) edge [->] node {$\sfrac{2}{3}$} (v1)
edge [->,bend left] node {$\sfrac{1}{3}$} (v2)
;
\end{tikzpicture}
\end{figure}
\textbf{Start:} Biblioteka, czyli:
\begin{description}
\item[$X_0$] miejsce początkowe
\item[$x_1$] miejsce w pierwszym kroku
\item[$x_2$] miejsce w drugim kroku
\end{description}
$\mathsf{Pr}(X_0=B)=1$
$$\begin{array}{l|lll}
& B & C & W \\ \hline
0 & 1 & 0 & 0 \\ \hline
1 & \sfrac{1}{2} & \sfrac{1}{4} & \sfrac{1}{4} \\ \hline
2 & \sfrac{7}{12} & \sfrac{5}{24} & \sfrac{5}{24} \\ \hline
\end{array}$$
$\mathsf{Pr}(X_2=B)=\frac{1}{2}*\frac{1}{2}+\frac{1}{4}*\frac{2}{3}+\frac{1}{4}*\frac{2}{3}=\frac{7}{12}$
\begin{description}
\item[$p_{ij}(x)$] $\rightarrow$ prawdopodobieństwo przejścia w jednym kroku z obiektu $i$ do obiektu $j$ w $x$ krokach.
\end{description}
Dla przykładu wyżej: $p_{BB}(2)=p_{BB}*p_{BB}+p_{BW}*p_{WB}+p_{BC}*p_{CB}$ 
$$\mathbb{P}= \bordermatrix{~ & B & C & W \cr
B & \sfrac{1}{2} & \sfrac{1}{4} & \sfrac{1}{4} \cr
C & \sfrac{2}{3} & 0 & \sfrac{1}{3}\cr
W & \sfrac{2}{3} & \sfrac{1}{3} &0 \cr}$$
\end{example*}

\subsection{Definicje}
\begin{definition}[Macierz stochastyczna]
Macierz $A=[a_{ij}]$ rozmiaru $n\times n$ nazywamy \textbf{macierzą stochastyczną} jeśli:
\begin{align*}
a_{ij}\geq 0 \text{ dla każdego }i,j\\
\sum_j a_{ij}=1 \text{ dla każdego } i 
\end{align*}
\end{definition}
\begin{definition}[(nieformalna) Łańcuchy Markowa]
Proces losowy, którego pozycja w następnym kroku zależy od obecnej (teraźniejszej) pozycji.
\end{definition}
\begin{definition}[(foramlna) Łańcuchy Markowa]
Niech $S$ będzie skończonym, niepustym zbiorem.\\
Niech macierz $\mathbb{P}=[p_{ij}]_{i,j\in S}$ będzie macierzą stochastyczną.\\
Niech $X_0, X_1, X_2,...$ będą zmiennymi losowymi: $X: \Omega \rightarrow S$.\\
Ciąg $$(X_i)_{i=0}^\infty$$ nazywamy \textbf{Łańcuchem Markowa} macierz przejścia $\mathbb{P}$, gdy:
\begin{enumerate}[label=\arabic*.]
\item Warunek / Własność Markowa:
\begin{align*}
&\mathsf{Pr}(X_t=s_t|X_{t-1}=s_{t-1},X_{t-2}=s_{t-2},...,X_0=s_0)=\\
&\mathsf{Pr}(X_t=s_t|X_{t-1}=s_{t-1})
\end{align*} dla $t\in \mathbb{N}_+$ i $s_t, s_{t-1},...,s_0\in S$\footnote{O ile mają sens, czytaj: $\mathsf{Pr}(X_t=s_t|X_{t-1}=s_{t-1},...,X_0=s_0)>0$}
\item $$\mathsf{Pr}(X_t=s_t|X_{t-1}=s')=p_{ss'}$$ dla $t\in \mathbb{N}_+$ i $s,s'\in S$\footnote{o ile $\mathsf{Pr}(X_{t-1}=s')>0$}
\end{enumerate}
Jest to definicja \textbf{Skończonego jednorodnego Łańcuchu Markowa} czytaj: $|S|<\infty$, gdzie 
\begin{description}
\item[$S$] to przestrzeń stanów Łańcuchów Markowa
\item[$X_0$] rozkład początkowy Łańcuchów Markowa
\end{description}
\end{definition}

\begin{example*} 
Ciąg dalszy ,,dylematu studenta'' z strony \pageref{exa:student}
\begin{align*}
&S=\{B,C,W\}\\
&X_i \rightarrow \text{ położenie (stan) Studenta po }i\text{ krokach}
&\mathbb{P}=\begin{bmatrix}
\sfrac{1}{2} & \sfrac{1}{4} & \sfrac{1}{4} \\
\sfrac{2}{3} & 0 & \sfrac{1}{3}\\
\sfrac{2}{3} & \sfrac{1}{3} &0 
\end{bmatrix}\\
&(X_i)_{i=0}^\infty \text{ jest Łańcuchem Markowa macierzy przejścia }\mathbb{P}
\end{align*}
\end{example*}
\begin{example*} 
Dwie osoby Ala i Franek rzucają na przemiennie monetą. Ala wygrywa, gdy nastąpi sekwencja $OOR$, Franek wygrywa, gdy nastąpi sekwencja $ROR$. 
\begin{itemize}
\item[] Kto ma większe szanse na wygraną? 
\item[] Ile średnio czasu trwa gra?
\end{itemize}
Próbny rozwiązania:
\begin{enumerate}[label=Próba \Roman*.]
\item $S=\{O,R\}$
\begin{description}
\item[$X_i$] co wypadło w $i$ tym rzucie
\item[$(X_i)_{i=0}^\infty$] Łańcuch Markowa o macierzy:
$$\mathbb{P}=\begin{bmatrix}
\sfrac{1}{2}&\sfrac{1}{2}\\\sfrac{1}{2}&\sfrac{1}{2}
\end{bmatrix}$$
\end{description}
\textbf{PUDŁO} - nie oddaje ducha gry

\item $S=\{O,R,W_A,W_F\}$
\begin{align*}
&X_i=\left\{\begin{matrix}
O/R & \text{ Nie ma zwycięzcy }\\
W_A & \text{ Wygrała Ala w }i\text{ tym rzucie} \\
W_F & \text{ Wygrał Franek w }i\text{ tym rzucie} \\
\end{matrix}
\right.\\
&(X_i)_{i=0}^\infty \text{to nie Łańcuch Markowa, gdyż: }\\
&\mathsf{Pr}(X_3=O|X_2=R,X_1=O,X_0=R)=0\\
&\mathsf{Pr}(X_3=O|X_2=R,X_1=R,X_0=R)=0
\end{align*}

\item $S=\{O,R,OO,RO,W_A,W_F\}$\\
$X_i$ to stan w którym jest gra po $i$ rzutach
\begin{figure}[H]
\centering
\begin{tikzpicture}[shorten >=1pt, auto, node distance=3cm, ultra thick,main node/.style={circle,draw,minimum size=.4cm,inner sep=0pt}]
\begin{scope}[every node/.style={font=\sffamily\large\bfseries}]
\node [main node](v1) at (0,0) {R};
\node [main node](v2) at (2,0) {RO};
\node [main node](v3) at (2,2) {OO};
\node [main node](v4) at (0,2) {O};
\node [main node](v5) at (4,0) {$W_F$};
\node [main node](v6) at (4,2) {$W_A$};
\end{scope}
\path
(v1) edge [->] node {$\sfrac{1}{2}$} (v2)
edge [loop below]   node {$\sfrac{1}{2}$} (v1)
(v2) edge [->] node {$\sfrac{1}{2}$} (v3)
edge [->] node {$\sfrac{1}{2}$} (v5)
(v3) edge [loop above] node {$\sfrac{1}{2}$} (v3)
edge [->] node {$\sfrac{1}{2}$} (v6)
(v4) edge [->] node {$\sfrac{1}{2}$} (v3)
edge [->] node {$\sfrac{1}{2}$} (v1)
;
\end{tikzpicture}
\end{figure}
To jest Łańcuch Markowa\footnote{Moim zadaniem Franek powinien się zbuntować gra serio NOT FAIR}
\end{enumerate}
\end{example*}

\begin{fact*}
$\mathsf{Pr}(t)$ opisuje macierz $\mathbb{P}^t$
\end{fact*}

\begin{example*}
Ciąg dalszy studenta ze strony \pageref{exa:student}.
\begin{align*}
&\mathbb{P}=\begin{bmatrix}
\sfrac{1}{2} & \sfrac{1}{4} & \sfrac{1}{4} \\
\sfrac{2}{3} & 0 & \sfrac{1}{3}\\
\sfrac{2}{3} & \sfrac{1}{3} &0 
\end{bmatrix}\\
&\mathbb{P}^2=\begin{bmatrix}
\sfrac{1}{2} & \sfrac{1}{4} & \sfrac{1}{4} \\
\sfrac{2}{3} & 0 & \sfrac{1}{3}\\
\sfrac{2}{3} & \sfrac{1}{3} &0 
\end{bmatrix}\begin{bmatrix}
\sfrac{1}{2} & \sfrac{1}{4} & \sfrac{1}{4} \\
\sfrac{2}{3} & 0 & \sfrac{1}{3}\\
\sfrac{2}{3} & \sfrac{1}{3} &0 
\end{bmatrix}=\begin{bmatrix}
\sfrac{42}{72} & \sfrac{15}{72} & \sfrac{15}{72} \\
\sfrac{40}{72} & \sfrac{20}{72} & \sfrac{12}{72}\\
\sfrac{40}{72} & \sfrac{12}{72} & \sfrac{20}{72}
\end{bmatrix}\\
&\bar{p}(1)=\begin{bmatrix}
\sfrac{1}{2} & \sfrac{1}{4} & \sfrac{1}{4}
\end{bmatrix}\\
&\bar{p}(2)=\begin{bmatrix}
\sfrac{42}{72} & \sfrac{15}{72} & \sfrac{15}{72}
\end{bmatrix}
\end{align*}
\end{example*}

Dla $(X_i)_{i=0}^\infty$ $\bar{p}(t)$ to wektor opisujący rozkład zmiennej po $t$ krokach.

\begin{fact*}
$\mathbb{P}$ i rozkład zmiennej losowej początkowej ($X_0$) jednoznacznie określa rozkład $X_t$\\
\textbf{Przepis:}
\begin{enumerate}[label=\arabic*.]
\item $\bar{p}(t)=\bar{p}(0)*\mathbb{P}^t$
\item $\bar{p}(t)=\bar{p}(t-1)*\mathbb{P}$
\end{enumerate}
\end{fact*}


\end{document}
